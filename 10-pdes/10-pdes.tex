% Compile with XeLaTeX, TeXLive 2013 or more recent
\documentclass{beamer}

% Base packages
\usepackage{fontspec}
\usepackage{xunicode}
\usepackage{xltxtra}

\usepackage{amsfonts}
\usepackage{amsmath}
\usepackage{longtable}
\usepackage{csquotes}
\usepackage{standalone}

\usepackage{mdwlist} % compact itemize lists environment

% Setup fonts
\newfontfamily\russianfont{CMU Serif}
\setromanfont{CMU Serif}
\setsansfont{CMU Sans Serif}
\setmonofont{CMU Typewriter Text}

% Setup Russian hyphenation. NOTE: this declaration *must* come after fontspec's font declarations,
% or a mysterious (but harmless in other respects) error "Improper `at' size (0.0pt), replaced by 10pt." would appear.
\usepackage{polyglossia}
\defaultfontfeatures{Scale=MatchLowercase, Mapping=tex-text}

\setdefaultlanguage[spelling=modern]{russian} % for polyglossia
\setotherlanguage{english} % for polyglossia
% \usepackage{libertine}

% Vector drawings 
\usepackage{tikz}
\usetikzlibrary{shapes, calc, arrows, fit, positioning, decorations, patterns, decorations.pathreplacing, chains, snakes}

% Be able to insert hyperlinks
\usepackage{hyperref}
\hypersetup{colorlinks=true, linkcolor=black, filecolor=black, citecolor=black, urlcolor=blue , pdfauthor=Grigory Rechistov <grigory.rechistov@phystech.edu>, pdftitle=Программное моделирование вычислительных систем}
% \usepackage{url}

% Misc optional packages
\usepackage{underscore}
\usepackage{amsthm}

% A new command to mark not done places
\newcommand{\todo}[1][Напиши меня]{{\color{red}TODO\ #1}}

\title{Параллельная симуляция. Параллельная дискретная симуляция}
\subtitle{Курс «Программное моделирование вычислительных систем»}
\subject{Курс «Программное моделирование вычислительных систем»}

\author[]{Григорий Речистов \\ \small{\href{mailto:grigory.rechistov@phystech.edu}{grigory.rechistov@phystech.edu}}}
\date{\today}
\pgfdeclareimage[height=0.5cm]{ilab-logo}{../ilab-noletters.png}
\logo{\pgfuseimage{ilab-logo}}

\typeout{Copyright 2015 Grigory Rechistov}

\usetheme{Berlin}
\setbeamertemplate{navigation symbols}{}%remove navigation symbols

\begin{document}

\begin{frame}
\titlepage
\end{frame}

\begin{frame}
\tableofcontents
\end{frame} 

\section{Обзор}

\begin{frame}{На прошлых лекциях}
\begin{itemize}
\item Моделирование многих агентов, работающих асинхронно — DES в один поток
\item Параллельное моделирование исполняющих устройств — процессоров
\end{itemize}

\end{frame}

\begin{frame}{Вопросы}
\begin{itemize}
\item Что такое атомарная инструкция?\pause
\item Что такое инструкция-барьер?\pause
\item Верно ли, что любая атомарная инструкция является барьером?
\end{itemize}

\end{frame}


\begin{frame}{Общая схема моделируемой системы}

\centering
\begin{tikzpicture}[font=\small]
    \node {TODO: a mess};
\end{tikzpicture}

\end{frame}

\section{Идея PDES}

\begin{frame}{DES}

\begin{tikzpicture}[>=latex, font=\scriptsize]
    \draw[->] (0,0) -- (10,0) node[pos=0.9, below] (sim-time) {Время};

    \foreach \x in { 1, 2, 3, 4, 5, 6, 7, 8, 9} {
        \draw (\x,-0.15) -- (\x,0.15) node (tick\x) {};
    };
    \node[shape=dart, draw, shape border rotate=270 ] at (2, 0.5) (currevent) {};
    \node[draw, arrow box, arrow box arrows={north:.7cm}, text width=2.5cm, align=center, below = 0cm of currevent] (tsim) {Текущее симулируемое время};
    \node[align=center, above=0.2cm of currevent, text width=2cm]  {Обрабатываемое событие};

    \node[shape=dart, draw, shape border rotate=270 ] at (5, 0.5) (futureevent) {};
    \node[align=center, above=0.2cm of futureevent, text width=2cm] {Запланированное событие};

    \node[fill=black!10, shape=dart, draw, shape border rotate=270 ] at (7, 0.5) (newbornevent) {};
    \node[align=center, above=0.2cm of newbornevent, text width=1.5cm] {Новое событие};

    \node[shape=dart, draw, shape border rotate=270 ] at (9, 0.5) (futureevent2) {};
    \node[shape=dart, draw, shape border rotate=270, above=0cm of futureevent2 ] (futureevent3) {};

    \draw[dashed, ->] (currevent.south) .. controls +(1,-0.5) and +(-1,-0.5) .. (newbornevent.south); % node[below, pos=0.7, text width=3cm] {Обработка события порождает новое событие в будущем};

\end{tikzpicture}

\end{frame}

\begin{frame}{Наивный PDES}
\centering

\input{../../simbook/metoda/drawings/two-queues-2}
\end{frame}

\begin{frame}{Проблемы}
\begin{itemize}
\item Нарушение каузальности (причинно-следственной связи)
\item Недетерминизм модели
\item Эффект ускорения от параллелизации не гарантирован
\end{itemize}

\end{frame}

\begin{frame}{Как детектировать нарушения}

\begin{itemize}
\item При пересылке сообщения добавлять к нему метку локального симулируемого времени
\item По получении проверять метку и корректировать точку создания события
\item При отрицательном значении корректировки — бить тревогу
\end{itemize}

\end{frame}

\begin{frame}{Метки времени}
\centering
\input{../../simbook/metoda/drawings/timestamps}
\end{frame}

\begin{frame}{Консервативно или оптимистично?}

Теперь мы можем обнаруживать
нарушения, но всё ещё не можем
избавиться от их появления/последствий.
Необходимо:
1. Предотвращать их возникновение
или
2. Подавлять их вредное проявление

\end{frame}

\begin{frame}{Консервативная схема 1}

●
 При посылке сообщения блокировать
отправителя до тех пор, пока получатель не
обработает связанного события
●
 Не даём «быстрым» потокам продвигаться
через этапы коммуникации
●
 Фактически вводим упорядоченность ==
последовательность при коммуникациях

\end{frame}

\begin{frame}{}
\centering
\input{../../simbook/metoda/drawings/send-and-block}    
\end{frame}

\begin{frame}{Взаимоблокировка}
\centering
\input{../../simbook/metoda/drawings/deadlock}

\end{frame}

\begin{frame}{}
Консервативная схема 3
●
 Необходимо детектировать ситуацию
дедлока и разрушать его, освобождая
один поток
●
 Лучший выбор — очередь с наименьшим
значением симулируемого времени
●
 Система может оказаться в ситуации,
когда большую часть времени почти все
потоки стоят => выигрыша в скорости нет

\end{frame}

\begin{frame}{}
Пустые сообщения 1
●
 Можно ли избежать блокировок?
●
 Необходимость в них возникает из-за того,
что отдельные потоки не знают, в какой
стадии находятся остальные
●
 Как поток A может узнать локальное время
потока B? Через временную метку,
хранящуюся в каждом событии

\end{frame}

\begin{frame}{Пустые сообщения}
Пустые сообщения 2
●
 Периодическая рассылка пустых (null)
сообщений, не связанных с
архитектурными событиями, но несущими
метку времени
●
 Теперь каждый поток имеет представление
о том, не слишком ли он далеко убежал в
будущее, и может сам
притормаживать/блокировать своё
исполнение

\end{frame}

\begin{frame}{Slack}

    \begin{tikzpicture}[>=latex, font=\scriptsize, align=left]
    \draw[->] (0,0) -- (10,0) node[pos=0.95, below] (sim-time1) {Время} node[pos=0., below] {Поток 1};
    \foreach \x in { 1, 2, 3, 4, 5, 6, 7, 8, 9} {
        \draw (\x,-0.15) -- (\x,0.15) node (tick\x) {};
    };
    \node[draw, arrow box, arrow box arrows={north:.7cm}] at (2, -1) {$T_1$};
    \node[shape=dart, draw, shape border rotate=270 ] at (2, 0.5)  (ev11) {};

    \draw[->] (0,-2) -- (10,-2) node[pos=0.95, below] (sim-time2) {Время} node[pos=0., below] {Поток 2};
    \foreach \x in { 1, 2, 3, 4, 5, 6, 7, 8, 9} {
        \draw (\x,-2.15) -- (\x,-1.85) node (tick\x) {};
    };
    \node[draw, arrow box, arrow box arrows={north:.7cm}] at (3, -3) {$T_2$};
    \node[shape=dart, draw, shape border rotate=270 ] at (3, -1.5)  (ev21) {};

    \draw[->] (0,-4) -- (10,-4) node[pos=0.95, below] (sim-time3) {Время} node[pos=0., below] {Поток 3};
    \foreach \x in { 1, 2, 3, 4, 5, 6, 7, 8, 9} {
        \draw (\x,-4.15) -- (\x,-3.85) node (tick\x) {};
    };
    \node[draw, arrow box, arrow box arrows={north:.7cm}] at (7, -5.2) {$T_3$\\блокировано};
    \node[shape=dart, draw, shape border rotate=270 ] at (8, -3.5)  (ev31) {};

%     \node[shape=dart, draw, shape border rotate=270 ] at (9, 0.5)  (ev12) {};
%     \node[shape=dart, draw, shape border rotate=270 ] at (6, -1.5)  (ev22) {};
%     \node[shape=dart, draw, shape border rotate=270 ] at (7, -3.5)  (ev32) {};

    \draw[->, dashed] (1,0) .. controls +(0.5, -1) and +(0.5, 1) .. (1,-4) node[yshift=-0.5cm] {null\\msg\textsubscript{1}} node[yshift=-0.5cm, xshift=1cm] {null\\msg\textsubscript{2}};
    \draw[->, dashed] (3,-2) .. controls +(1, -0.5) and +(1, 0.5) .. (3,-3.5) node[pos=0.5, right] {null msg\textsubscript{2} (в полёте)};

    \draw[<->] (1, -6) -- (7, -6) node[midway, above] {$T_3 - T_{msg1} \geqslant MaxSlack$};
    
    \end{tikzpicture}
\end{frame}

\begin{frame}{Пустые сообщения}

Как часто рассылать сообщения?
●
 Часто => потоки могут бежать свободнее, но
большой трафик
●
 Редко => потоки не имеют актуальной информации
о состояниях остальных и простаивают зря
Кому рассылать?
●
 Всем остальным — большой трафик
●
 Не всем — дедлоки вероятны
●
 Случайным адресатам — можно балансировать

\end{frame}

\begin{frame}{Частный случай: домены синхронизации}
    \centering
\input{../../simbook/metoda/drawings/send-and-block}   
\end{frame}

\begin{frame}{Оптимистичные схемы}
Даём параллельной программе работать самой по
себе
Периодически сохраняем (корректное) состояние
всей системы
При обнаружении каузальных ошибок
откатываемся до ближайшего сохранённого
состояния
Проходим проблемный участок аккуратными
методами (напр. консервативно)

\end{frame}

\begin{frame}{Точки сохранения}
\centering
\input{../../simbook/metoda/drawings/checkpoints} 
\end{frame}

\begin{frame}{Time Warp}

Сообщение — набор данных, описывающих событие, которое
должно быть добавлено в одну из очередей событий. Оно
характеризуется, кроме своего непосредственного
содержимого, виртуальными временами отправки tsend и
обработки treceive.
LVT (local virtual time) — значение симулируемого времени
отдельного потока, участвующего в симуляции. Для
создаваемых событий их время отправки tsend равно
значению LVT отправителя. В отличие от консервативных
схем, эта величина может как расти в процессе симуляции,
так и убывать в случае отката процесса.
\end{frame}


\begin{frame}{GVT}
 GVT (global virtual time) — глобальное время для всей
симуляции, определяющее, до какой степени возможно её
откатывать. Глобальное время всегда монотонно растёт, всегда
оставаясь позади локального времени самого медленного
потока, а также оно меньше времени отправки самого раннего
ещё не доставленного события:
$$GVT \leq \min \left( \min\limits_{i} LVT_{i}, \min\limits_{k} t^{send}_k  \right).$$
\end{frame}


\begin{frame}{Straggler, antimessage}

Отставшее сообщение (straggler) — событие, пришедшее
в очередь с меткой времени treceivestraggler, меньшей,
чем LVT получателя. Его обнаружение вызывает откат
текущего состояния, при этом LVT уменьшается, пока не
станет меньше, чем treceivestraggler, после чего оно
может быть обработано. После этого возобновляется
прямая симуляция.
Антисообщение (antimessage). Каждое антисообщение
соответствует одному ранее созданному сообщению,
порождённому в интервале симулируемого времени
[tstraggler, LVTi] и вызывает эффект, обратный его
обработке (т.е. возвращает состояние в исходное).

\end{frame}


\begin{frame}{Работа Time Warp}
\centering
\input{../../simbook/metoda/drawings/time-warp} 
\end{frame}

\begin{frame}{Fossil Collection}
Освобождение места, занятого сообщениями, расположенными левее GVT
\end{frame}

\begin{frame}{}

\end{frame}

% \begin{frame}{}
% 
% \end{frame}


\begin{frame}[allowframebreaks]{Рекомендуемая литература}
\begin{thebibliography}{99}
\bibitem{fujimoto} Fujimoto Richard M. Parallel discrete event simulation // Commun. ACM. — 1990. — Окт. — Т. 33, No 10. — С. 30– 53. \url{http://doi.acm.org/10.1145/84537.84545}
\bibitem{liu} Liu Jason. Parallel Discrete-Event Simulation. — 2009. \url{http://www.cis.fiu.edu/~liux/research/papers/pdes-eorms09.pdf}
\bibitem{misra}Jayadev Misrsa. Distributed discrete-event simulation //ACM Computing Surveys 18 1986 \url{www.cis.udel.edu/~cshen/861/notes/p39-misra.pdf}

% \bibitem{simos} Lantz Robert E. Parallel SimOS: Performance and Scalability for Large System Simulation. — 2007 \url{http://cs.stanford.edu/~rlantz/papers/lantz-thesis.pdf}

\end{thebibliography}
\end{frame}


\begin{frame}{На следующей лекции}
\centering

Эффективная современная виртуализация

\end{frame}


\begin{frame}

{\huge{Спасибо за внимание!}\par}

\vfill

Слайды и материалы курса доступны по адресу \url{http://is.gd/ivuboc} % http://atakua.doesntexist.org/wordpress/simulation-course-russian/

\vfill

\tiny{\textit{Замечание}: все торговые марки и логотипы, использованные в данном материале, являются собственностью их владельцев. Представленная здесь точка зрения отражает личное мнение автора, не выступающего от лица какой-либо организации.}

\end{frame}

\end{document}

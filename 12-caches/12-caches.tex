% Compile with XeLaTeX, TeXLive 2013 or more recent
\documentclass{beamer}

% Base packages
\usepackage{fontspec}
\usepackage{xunicode}
\usepackage{xltxtra}

\usepackage{amsfonts}
\usepackage{amsmath}
\usepackage{longtable}
\usepackage{csquotes}
\usepackage{standalone}

% Setup fonts
\newfontfamily\russianfont{CMU Serif}
\setromanfont{CMU Serif}
\setsansfont{CMU Sans Serif}
\setmonofont{CMU Typewriter Text}

% Setup Russian hyphenation. NOTE: this declaration *must* come after fontspec's font declarations,
% or a mysterious (but harmless in other respects) error "Improper `at' size (0.0pt), replaced by 10pt." would appear.
\usepackage{polyglossia}
\defaultfontfeatures{Scale=MatchLowercase, Mapping=tex-text}

\setdefaultlanguage[spelling=modern]{russian} % for polyglossia
\setotherlanguage{english} % for polyglossia

% Vector drawings
\usepackage{tikz}
\usetikzlibrary{shapes, calc, arrows, fit, positioning, decorations, patterns, decorations.pathreplacing, chains, snakes}
\usepackage[siunitx]{circuitikz}

% Be able to insert hyperlinks
\usepackage{hyperref}
\hypersetup{colorlinks=true, linkcolor=black, filecolor=black, citecolor=black, urlcolor=blue , pdfauthor=Grigory Rechistov <grigory.rechistov@phystech.edu>, pdftitle=Курс «Программное моделирование вычислительных систем»}
% \usepackage{url}

% Misc optional packages
\usepackage{underscore}
\usepackage{amsthm}

% A new command to mark not done places
\newcommand{\todo}[1][Напиши меня]{{\color{red}TODO\ #1}}

\title{Роль моделирования в процессе разработки программно-аппаратных платформ}
\subtitle{Курс «Программное моделирование вычислительных систем»}
\subject{Курс «Программное моделирование вычислительных систем»}

\author[Григорий Речистов]{Григорий Речистов \\ \small{\href{mailto:grigory.rechistov@phystech.edu}{grigory.rechistov@phystech.edu}}}
\date{\today}
\pgfdeclareimage[height=0.5cm]{ilab-logo}{../ilab-noletters.png}
\logo{\pgfuseimage{ilab-logo}}

\typeout{Copyright 2014 Grigory Rechistov}

\usetheme{Berlin}
\setbeamertemplate{navigation symbols}{}%remove navigation symbols

\begin{document}

\begin{frame}
    \maketitle
\end{frame}

\begin{frame}
    \tableofcontents
\end{frame}

\section{История использования}

\begin{frame}{Атомарные инструкции}

\begin{tikzpicture}[>=latex]

\node[draw, circle] (core1) {Ядро 1};

\node[draw, circle, right = of core1] (core2) {Ядро 2};

\node[draw, circle, right = of core2] (core3) {Ядро 3};

\node[draw, circle, right = of core3] (core4) {Ядро 4};

\node[below = 2cm of core1] (c1) {};
\node[below = 2cm of core4] (c2) {};

\node[draw, fit = (c1) (c2) ] (shmem) {Общая память} ;

\path (core2.south) -- (core3.south) node[midway, yshift = -1cm] (lock) {\texttt{LOCK}};
\draw[thick,] (core2) |- (lock);
\draw[thick,->] (lock) -- (shmem);

\path (core1.south) |- (lock) coordinate[midway] (block1); %{X};
\path (core3.south) |- (lock) coordinate[midway] (block3); %{X};
\path (core4.south) |- (lock) coordinate[midway] (block4); %{X};

\draw[->] (core1.south) -- (block1);
\draw[->] (core3.south) -- (block3);
\draw[->] (core4.south) -- (block4);

\draw[dashed] (block1) -- (lock);
\draw[dashed] (lock)   -- (block3);
\draw[dashed] (block3) -- (block4);

\end{tikzpicture}

\end{frame}

\section{Литература}

\begin{frame}[allowframebreaks]{Литература}
\begin{thebibliography}{99}
    \bibitem{simbook} Программное моделирование вычислительных систем (лекции). \url{http://atakua.doesntexist.org/public/archive/simcourse/simulation-lectures-latest.pdf}

    \bibitem{practicum} Лабораторный практикум по программному моделированию.  \url{http://atakua.doesntexist.org/public/archive/simcourse/simulation-practicum-latest.pdf}

\end{thebibliography}
\end{frame}


\section{Конец}
% The final "thank you" frame
\begin{frame}

{\huge{Спасибо за внимание!}\par}

\vfill

Слайды и материалы курса доступны по адресу \url{http://is.gd/ivuboc} % http://atakua.doesntexist.org/wordpress/simulation-course-russian/

\vfill

\tiny{\textit{Замечание}: все торговые марки и логотипы, использованные в данном материале, являются собственностью их владельцев. Представленная точка зрения отражает личное мнение автора.
%Материалы доступны по лицензии Creative Commons Attribution-ShareAlike (Атрибуция — С сохранением условий) 4.0 весь мир (в т.ч. Россия и др.). Чтобы ознакомиться с экземпляром этой лицензии, посетите \url{http://creativecommons.org/licenses/by-sa/4.0/}
}

\end{frame}


\end{document}

% Compile with XeLaTeX, TeXLive 2013 or more recent
\documentclass{beamer}

% Base packages
\usepackage{fontspec}
\usepackage{xunicode}
\usepackage{xltxtra}

\usepackage{amsfonts}
\usepackage{amsmath}
\usepackage{longtable}
\usepackage{csquotes}
\usepackage{standalone}

% Setup fonts
\newfontfamily\russianfont{CMU Serif}
\setromanfont{CMU Serif}
\setsansfont{CMU Sans Serif}
\setmonofont{CMU Typewriter Text}

% Setup Russian hyphenation. NOTE: this declaration *must* come after fontspec's font declarations,
% or a mysterious (but harmless in other respects) error "Improper `at' size (0.0pt), replaced by 10pt." would appear.
\usepackage{polyglossia}
\defaultfontfeatures{Scale=MatchLowercase, Mapping=tex-text}

\setdefaultlanguage[spelling=modern]{russian} % for polyglossia
\setotherlanguage{english} % for polyglossia

% Vector drawings
\usepackage{tikz}
\usetikzlibrary{shapes, calc, arrows, fit, positioning, decorations, patterns, decorations.pathreplacing, chains, snakes}

\usepackage[siunitx]{circuitikz}

% Be able to insert hyperlinks
\usepackage{hyperref}
\hypersetup{colorlinks=true, linkcolor=black, filecolor=black, citecolor=black, urlcolor=blue , pdfauthor=Grigory Rechistov <grigory.rechistov@phystech.edu>, pdftitle=Курс «Программное моделирование вычислительных систем»}
% \usepackage{url}

% Misc optional packages
\usepackage{underscore}
\usepackage{amsthm}

% A new command to mark not done places
\newcommand{\todo}[1][Напиши меня]{{\color{red}TODO\ #1}}

\title{Исполняющие и неисполняющие модели. Симуляция многопроцессорных систем}
\subtitle{Курс «Программное моделирование вычислительных систем»}
\subject{Курс «Программное моделирование вычислительных систем»}

\author[Григорий Речистов]{Григорий Речистов \\ \small{\href{mailto:grigory.rechistov@phystech.edu}{grigory.rechistov@phystech.edu}}}
\date{\today}
\pgfdeclareimage[height=0.5cm]{ilab-logo}{../ilab-noletters.png}
\logo{\pgfuseimage{ilab-logo}}

\typeout{Copyright 2014 Grigory Rechistov}

\usetheme{Berlin}
\setbeamertemplate{navigation symbols}{}%remove navigation symbols

\begin{document}

\begin{frame}
    \maketitle
\end{frame}

\begin{frame}
    \tableofcontents
\end{frame}

\section{Два класса моделей}

\begin{frame}{На этой лекции}
\begin{itemize}
    \item Совместная работа с моделью процессора.
    \item Работа с несколькими процессорами сразу.
\end{itemize}
\end{frame}

\section{Косимуляция}

\begin{frame}{Какие типы моделей нам известны}

\begin{itemize}
    \item Интерпретация\pause: процессоры. \pause
    \item Очередь событий\pause: таймер.\pause
    \item «Мгновенная» модель «стимул — отклик»\pause: ОЗУ.
\end{itemize}

\end{frame}

\begin{frame}{Модель многопроцессорной системы}
\begin{tikzpicture}[>=latex]

\node[draw, circle] (core1) {Ядро 1};
\node[draw, circle, right = of core1] (core2) {Ядро 2};
\node[draw, circle, right = of core2] (core3) {Ядро 3};
\node[draw, circle, right = of core3] (core4) {Ядро $N$};

\coordinate[below = 2.3cm of core1] (c3);
\coordinate[below = 1.5cm of core4] (c4);

\node[draw, fit = (c3) (c4), inner ysep=1pt ] (shmem) {Общая память};

\draw[<->] (core1.south) -- (shmem);
\draw[<->] (core2.south) -- (shmem);
\draw[<->] (core3.south) -- (shmem);
\draw[<->] (core4.south) -- (shmem);

\end{tikzpicture}

\end{frame}

\begin{frame}[fragile]{Квотированная симуляция}
\begin{tikzpicture}[>=latex]
\draw[->] (0,0) -- (8,0) node[pos=0.9, below] (sim-time) {Физическое время};

\foreach \x in { 1, 2, 3, 4, 5, 6, 7} {
    \draw (\x,-0.15) -- (\x,0.15) node (tick\x) {};
};
\matrix[anchor=south west] at (-0.5,0.5){
    \node[] {CPU3}; & & & \node[shape=single arrow, draw, text width = 2cm, inner xsep = 0cm, fill=black!5] (arr3) {}; \\
    \node {CPU2}; & & \node[shape=single arrow, draw, text width = 2cm, inner xsep = 0cm, fill=black!10] (arr2) {}; & \\
    \node {CPU1}; & \node[shape=single arrow, draw, text width = 2cm, inner xsep = 0cm, fill=black!15] (arr1) {}; & & \\
};

\draw[->] (arr1.east) -- (arr2.west);
\draw[->] (arr2.east) -- (arr3.west);

\end{tikzpicture}

\end{frame}

\begin{frame}[fragile]{Квотированная симуляция}

    \begin{tikzpicture}[>=latex]
        \draw[->] (0,0) -- (8,0) node[pos=0.9, below] (sim-time) {Симулируемое время};

    \foreach \x in { 1, 2, 3, 4, 5, 6, 7} {
        \draw (\x,-0.15) -- (\x,0.15) node (tick\x) {};
    };
    \matrix[anchor=south west] at (-0.5,0.5){
        \node {CPU3}; & \node[shape=single arrow, draw, text width = 2cm, inner xsep = 0cm, fill=black!5] (arr3) {};  \\
        \node {CPU2}; & \node[shape=single arrow, draw, text width = 2cm, inner xsep = 0cm, fill=black!10] (arr2) {}; \\
        \node {CPU1}; & \node[shape=single arrow, draw, text width = 2cm, inner xsep = 0cm, fill=black!15] (arr1) {}; \\
    };

    % \draw[->] (arr1.east) -- (arr2.west);
    % \draw[->] (arr2.east) -- (arr3.west);

    \end{tikzpicture}

\end{frame}

\begin{frame}{Размер квоты}
\begin{itemize}
\item  Процессор может исполнить меньше инструкций, чем содержится в выданной ему квоте.
\item Не следует увлекаться излишне большими квотами, пытаясь ускорить исполнение.
\item В DES-модели могут быть реализованы псевдо-события, обработка которых вызывает переключение текущего процессора.
\end{itemize}

\end{frame}



\begin{frame}{Совместная симуляция DES и исполняющей модели}
\centering
    \begin{tikzpicture}[>=latex, font=\scriptsize]

    \draw[->] (-0.5,0) -- (10.5,0); % node[pos=0.9, above] (sim-time) {Время};

    \begin{scope}
    \clip (0,-2) rectangle (10, 2.5);
    \foreach \x in { 1, 2, 3, 4, 5, 6, 7, 8, 9} {
        \draw (\x,-0.15) -- (\x,0.15) node (tick\x) {};
    };

    \node[shape=dart, draw, shape border rotate=270 ] at (1, 0.5) (event1) {};
    \node[shape=dart, draw, shape border rotate=270 ] at (5, 0.5) (event2) {};
    \node[shape=dart, draw, shape border rotate=270 ] at (9, 0.5) (event3) {};

    \node[above of=event2] (desalabel) {Дискретные события};
    \draw[->] (desalabel) -- (event1);
    \draw[->] (desalabel) -- (event2);
    \draw[->] (desalabel) -- (event3);

    \draw (3,-0.5) ellipse[x radius = 2cm, y radius = 0.5cm] node {Исполнение процессора} ;
    \draw (7,-0.5) ellipse[x radius = 2cm, y radius = 0.5cm] node {Исполнение процессора} ;

    \draw (-1,-0.5) ellipse[x radius = 2cm, y radius = 0.5cm] node {} ;
    \draw (11,-0.5) ellipse[x radius = 2cm, y radius = 0.5cm] node {} ;
    \end{scope}
    \end{tikzpicture}


\end{frame}

\begin{frame}{Косимуляция}

\centering
    \begin{tikzpicture}[>=latex]
    \node[draw, circle, text width = 3cm, text badly centered] (dessim) {Симулятор дискретных событий};
    \node[draw, circle, text width = 3cm, text badly centered, right = 2.5cm of dessim] (execsim) {Модель исполняющего устройства};

    \draw (dessim.45)   edge[bend left = 45, ->] (execsim.135);
    \node[above=1cm of execsim.135] {\small Длительность до следующего события};
    \draw (execsim.225) edge[->, bend left = 45] (dessim.315);
    \node[below=1cm of dessim.315] {\small Число исполненных шагов};

    \end{tikzpicture}

\end{frame}



% \begin{frame}{}
% 
% 
% \end{frame}

\section{Практический пример}

\begin{frame}[fragile]{Пример на модели \texttt{viper-busybox.simics}}
Квота на восьмиядерной конфигурации:
\begin{verbatim}
simics> cpu-switch-time
Current time quantum:     0.0001 s
  200000.0  viper.mb.cpu0.core[0][0]
  200000.0  viper.mb.cpu0.core[0][1]
  200000.0  viper.mb.cpu0.core[1][0]
  200000.0  viper.mb.cpu0.core[1][1]
  200000.0  viper.mb.cpu0.core[2][0]
  200000.0  viper.mb.cpu0.core[2][1]
  200000.0  viper.mb.cpu0.core[3][0]
  200000.0  viper.mb.cpu0.core[3][1]
Default time quantum not set yet
\end{verbatim}

\end{frame}


\begin{frame}[fragile]{Пример на модели \texttt{viper-busybox.simics}}
Симулируемое время на восьмиядерной конфигурации:
\begin{verbatim}
simics> ptime -all
processor                      steps      cycles  time [s]
viper.mb.cpu0.core[0][0]  4602030000  5134030000     2.567
viper.mb.cpu0.core[0][1]  4925600000  5134000000     2.567
viper.mb.cpu0.core[1][0]  4925600000  5134000000     2.567
viper.mb.cpu0.core[1][1]  4925600000  5134000000     2.567
viper.mb.cpu0.core[2][0]  4925600000  5134000000     2.567
viper.mb.cpu0.core[2][1]  4925600000  5134000000     2.567
viper.mb.cpu0.core[3][0]  4925600000  5134000000     2.567
viper.mb.cpu0.core[3][1]  4925600000  5134000000     2.567
\end{verbatim}
    
\end{frame}


\section{Литература}

\begin{frame}[allowframebreaks]{Литература}
\begin{thebibliography}{99}
    \bibitem{simics-processor-integration} Wind River. Simics Processor Integration Guide.
\end{thebibliography}
\end{frame}


\section{Конец}
% The final "thank you" frame
\begin{frame}

{\huge{Спасибо за внимание!}\par}

\vfill

Слайды и материалы курса доступны по адресу \url{http://is.gd/ivuboc} % http://atakua.doesntexist.org/wordpress/simulation-course-russian/

\vfill

\tiny{\textit{Замечание}: все торговые марки и логотипы, использованные в данном материале, являются собственностью их владельцев. Представленная точка зрения отражает личное мнение автора.
Материалы доступны по лицензии Creative Commons Attribution-ShareAlike (Атрибуция — С сохранением условий) 4.0 весь мир (в т.ч. Россия и др.). Чтобы ознакомиться с экземпляром этой лицензии, посетите \url{http://creativecommons.org/licenses/by-sa/4.0/}
}

\end{frame}


\end{document}

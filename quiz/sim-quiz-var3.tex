% Compile with XeLaTeX only
%
% \documentclass[a4paper, addpoints, answers]{exam}
\documentclass[a4paper, addpoints]{exam}

\pointpoints{б.}{б.}
\bonuspointpoints{б.}{б.}

\def \variant {3}

\usepackage{fontspec}
%\usepackage[utf8]{inputenc} % uncomment for latex2rtf 

\usepackage{xunicode} % some extra unicode support
\usepackage{xltxtra}

\usepackage{amsfonts}
\usepackage{amsmath}

\usepackage{csquotes}

\usepackage{polyglossia}
\defaultfontfeatures{Scale=MatchLowercase, Mapping=tex-text}

\newfontfamily\russianfont{CMU Serif}
\setromanfont{CMU Serif}
\setsansfont{CMU Sans Serif}
\setmonofont{CMU Typewriter Text}

\setdefaultlanguage[spelling=modern]{russian}
\setotherlanguage{english}
\newcommand{\todo}{{\color{red}TODO}\ }
\usepackage{graphicx}

\tolerance=9999 % let the text underfull be ugly as hell, nobody cares.
\emergencystretch=3cm

\usepackage{hyperref}
\hypersetup{colorlinks=true, linkcolor=black, filecolor=black, citecolor=black, urlcolor=black , pdfauthor=Grigory Rechistov, pdftitle=Контрольная по курсу <<Основы программного моделирования ЭВМ>>}

\usepackage{footnpag}
\usepackage{indentfirst}
\usepackage{underscore}
\usepackage{url}

\usepackage{listings}
\lstset{basicstyle=\footnotesize\ttfamily, breaklines=true, keepspaces=true}

\usepackage{nameref}
\usepackage{amsthm}
\usepackage{enumitem} % continue enumeration 
\usepackage{subfigure}
\usepackage{mdwlist} % compact itemize lists environment

\setcounter{tocdepth}{2}

\renewcommand{\solutiontitle}{\noindent\textbf{Ответ:}\enspace}

\title{Основы программного моделирования ЭВМ. Контрольная работа. Вариант \textnumero \variant}
\author{}
\newcommand{\testdate}{\today}
\date{\testdate}

\typeout{Copyright 2012 Grigory Rechistov and the contributors.}
\begin{document}

\pagestyle{headandfoot}
\runningheadrule
\firstpageheader{Вариант \textnumero \variant}{Кафедра <<Микропроцессорные технологии>>}{\testdate}
\runningheader{Вариант \textnumero \variant}{Кафедра <<Микропроцессорные технологии>>}{\testdate}
\coverfooter{}{Стр. \thepage\ из \numpages}{}
\firstpagefooter{}{Стр. \thepage\ из \numpages}{}
\runningfooter{}{Стр. \thepage\ из \numpages}{}

\maketitle\thispagestyle{headandfoot} % a hack to have proper footers and headers on the first page as well

\medskip
\makebox[0.9\textwidth]{Ф.И.О.\enspace\hrulefill}
\medskip

\makebox[0.9\textwidth]{Группа\enspace\hrulefill}
\medskip

\makebox[0.9\textwidth]{e-mail\enspace\hrulefill}

% Grade tables
\hqword{Вопрос}
\hpword{Баллов}
\hsword{Результат}
\htword{Сумма}
\vqword{Вопрос}
\vpword{Баллов}
\vsword{Результат}
\vtword{Сумма}
\cellwidth{0.5em}
\gradetablestretch{1.8}
\begin{center}
\tiny
%\gradetable[h][questions]
\partialgradetable{multiplechoice}[h][questions]\\
\partialgradetable{fullquestions}[h][questions]\\
\partialgradetable{bonuswork}[h][questions]\\
\end{center}

\medskip

\begin{questions}

% Introduce several grading range so that grading tables are short enough to fit on the page

\begingradingrange{multiplechoice} 

\question[1] Какие из указанных ниже компонентов обязательны для реализации интерпретатора?
\begin{choices}
    \correctchoice декодер
    \choice дизассемблер
    \choice енкодер
    \correctchoice блоки реализации семантики отдельных инструкций
    \choice кэш декодированных инструкций
\end{choices}

\question[1] Какой тип инструкций наиболее сложен с точки зрения симуляции в режиме прямого исполнения?
\begin{choices}
    \choice арифметические
    \correctchoice привилегированные
    \choice с плавающей запятой
    \choice условные и безусловные переходы
\end{choices}

% \question[1] Чем чревата излишне частая отправка пустых (null) сообщений в консервативной схеме PDES с детектированием взаимоблокировок?
% \begin{solution}[1cm]
% Низкой скоростью работы симулятора из-за перегруженности хозяйской сети.
% \end{solution}
% 
% \question[1] Чем чревата недостаточно частая отправка пустых (null) сообщений в консервативной схеме PDES с детектированием взаимоблокировок?
% \begin{solution}[1cm]
% Низкой скоростью работы симулятора из-за частой блокировки потоков, слишком далеко убежавших в симулируемое будущее.
% \end{solution}

% \question[1] Выберите правильные продолжения фразы: Частая отправка пустых (null)  сообщений нежелательна, так как
% \begin{choices}
%     \correctchoice это может ограничивать скорость симуляции
%     \choice это может вызвать нарушение каузальности симуляции
%     \choice это может привести к взаимоблокировке потоков
%     \choice это может привести к переполнению очередей сообщений
% \end{choices}


\question[1] Выберите правильные варианты завершения фразы: в потактовых моделях, созданных на основе концепции портов
\begin{choices}
    \correctchoice функциональные модули не имеют внутреннего состояния
    \choice функциональные модули могут иметь внутреннее состояние
    \choice функциональные модули всегда имеют внутреннее состояние
\end{choices}
    
\question[1] Выберите правильные варианты завершения фразы: в потактовых моделях, созданных на основе концепции портов
\begin{choices}
    \correctchoice ширина входа и выхода порта должны быть равны
    \choice ширина входа и выхода порта могут различаться
    \choice количество выходов функционального элемента должно быть равно единице
    \choice количество входов и выходов функционального элемента должно совпадать
\end{choices}
    
\question[1] Выберите правильные варианты окончания фразы: наличие единственного \texttt{switch} для всех гостевых инструкций в коде интерпретатора
\begin{choices}
    \choice увеличивает его скорость по сравнению со схемой сцепленной интерпретации
    \choice упрощает его алгоритмическую структуру по сравнению со схемой сцепленной интерпретации
    \correctchoice уменьшает его скорость по сравнению со схемой сцепленной интерпретации
    \choice не влияет на скорость работы интерпретатора
\end{choices}

\question[1] Выберите правильный вариант окончания фразы: в моделях DES новые события могут быть добавлены
\begin{choices}
    \choice только к голове очереди событий
    \choice только к хвосту очереди событий
    \correctchoice в любую позицию в очереди событий
\end{choices}

\question[1] Какой байт будет расположен первым в памяти на Big Endian системе при записи числа \texttt{0xbaadc0de} в память?
\begin{solution}[1cm]
0xba
\end{solution}

\question[1] Выберите правильные варианты продолжения фразы: использование кэшей при работе приложения целесообразно, если
\begin{choices}
    \correctchoice    программа показывает временн\'{у}ю локальность доступов
    \choice программа не обращается в оперативную память
    \choice программа работает с очень большим объёмом данных
    \correctchoice программа показывает пространственную локальность доступов
    \choice программа работает с объёмом данных, меньшим ёмкости кэша
\end{choices}

\question[1] Выберите правильное отношение для фразы <<машинное слово длиной $w$ байт выровнено в памяти по адресу $addr$>>.
\begin{choices}
	\choice $addr \neq 0 (\mod w)$
	\choice $w = 2^k$ и $addr = 2^m$, $k,m \in \mathbb{N}$
    \correctchoice $addr = 0 (\mod w)$
\end{choices}

\question[1] Выберите сценарии, когда скорость симуляции, превышающая скорость работы реальной системы, нежелательна.
\begin{choices}
    \choice программа вычисляет значение некоторой функции в узлах сетки и выводит результаты на экран.
    \correctchoice система ожидает ввода пользователя в течение ограниченного времени.
    \choice программа взаимодействует по моделируемой сети с другой моделируемой системой.
\end{choices}

\question[1] Какой вид активности невозможен при симуляции трасс?
\begin{choices}
\correctchoice Интерактивное взаимодействие с пользователем.
\choice Загрузка операционной системы.
\choice Работа с периферийными устройствами.
\end{choices}

\question[1] Какая инструкция для архитектуры IA-32 не может быть использована как волшебная?
\begin{choices}
\choice \texttt{CPUID} --- идентификация процессора.
\correctchoice \texttt{INT} --- привилегированное программное прерывание.
\choice \texttt{NOP} --- пустая операция.
\end{choices}

\endgradingrange{multiplechoice}

\begingradingrange{fullquestions}

\question[3] Дайте определение понятия <<функциональный симулятор>>
\begin{solution}[2cm]
Модель, обеспечивающая корректное выполнение алгоритмов отдельных инструкций, но при этом не гарантирующая корректность симулируемых длительностей операций.
\end{solution}
    
% \question[3] Дайте определение понятия <<гибридный симулятор>>
% \begin{solution}[2cm]
% Модель, частично реализованная в программе для обычного компьютера, а частично --- на специализированном оборудовании, например, на ПЛИС.
% \end{solution}

\question[3] Опишите, что происходит на стадии \textbf{Fetch} работы процессора.
\begin{solution}[1cm]
Чтение из памяти машинного кода, соответствующего текущей инструкции.
\end{solution}

% \question[1] Почему наивное использование direct execution в симуляторе невозможно?
% \begin{solution}[1cm]
% Решение
% \end{solution}


\question[3] Какой вид программ обычно исполняется в привилегированном режиме процессора?
\begin{solution}[1cm]
Операционные системы, мониторы виртуальных машин первого типа.
\end{solution}

\question[3] Дайте определение понятию <<байт>>.
\begin{solution}[1cm]
Минимальная адресуемая в данной архитектуре единица информации.
\end{solution}

\question[3] Дайте определение понятию <<капсула>>, используемому в двоичной трансляции.
\begin{solution}[1cm]
Блок хозяйского машинного кода, моделирующий одну конкретную гостевую инструкцию.
\end{solution}
    
\question[3] Дайте определение понятию FLOPS, используемой для измерения скорости программ.
\begin{solution}[1cm]
Количество арифметических операций над числами с плавающей запятой, выполняемых за одну секунду.
\end{solution}

\question[3] Дайте определение понятию <<неисполняющее устройство>>
\begin{solution}[2cm]
Модель, изменения в состоянии которой происходят через интервалы, равные нескольким шагам симулируемого времени.
\end{solution}

\question[3] Что такое привилегированный режим процессора? Какие приложения обычно в нём исполняются?
\begin{solution}[2cm]
Режим процессора, в котором допустимы дополнительные инструкции, способные изменять глобальное состояние всей ЭВМ. В этом режиме обычно исполняются операционные системы.
\end{solution}

\question[3] Дайте определение понятию <<префиксный код>>
\begin{solution}[1cm]
Код со словом переменной длины, имеющий свойство: если в код входит слово \textit{a}, то для любой непустой строки \texttt{b} слова \textit{ab} в коде не существует. 
\end{solution}

\question[3] Сформулируйте закон Литтла
\begin{solution}[2cm]
Среднее число $N$ клиентов за достаточно долгосрочный период в устойчиво функционирующей системе  равно средней норме или скорости прибытия, умноженной на определённое за тот же период среднее время $T$, которое один клиент проводит в системе: $N = \lambda T$
\end{solution}

\question[3] Дайте определение понятию <<квота>>. используемому в симуляции многопроцессорных систем
\begin{solution}[1cm]
Максимальное количество шагов, симулируемых одним процессором без переключения на симуляцию других.
\end{solution}

\question[3] Дайте определение гипервизора второго типа.
\begin{solution}[2cm]
Гипервизоры второго типа  не заменяют операционную систему, но работают поверх её как обычное пользовательское приложение.
\end{solution}

\endgradingrange{fullquestions} 

\begingradingrange{morequestions}
    
% \question[1] Выберите правильные составляющие задачи <<code discovery>> (обнаружение кода) в ДТ.
% \begin{choices}
%     \choice поиск кода внутри исполняемого файла
%     \correctchoice поиск границ инструкций при работе двоичного транслятора
%     \choice     поиск границ инструкций при работе интерпретатора
%     \correctchoice различение гостевого кода от гостевых данных
%     \choice     декодирование гостевых инструкций
%     \choice поиск некорректных гостевых инструкций
% \end{choices}

% \question[1] Какая из следующих типов ситуаций при исполнении процессора является асинхронной по отношению к работе текущей инструкции?
% \begin{choices}
%     \correctchoice прерывание (interrupt)
%     \choice ловушка (trap)
%     \choice исключение (exception)
%     \choice промах (fault)
% \end{choices}

% \question[1] Почему не будет работать \textbf{наивная} схема параллельного DES? Выберите верные ответы
% \begin{choices}
%     \correctchoice Недетерминизм модели
%     \choice Невозможно организовать передачу сообщений между потоками
%     \correctchoice Возможно нарушение каузальности
%     \choice Невозможно подобрать точно квоту выполнения
% \end{choices}
    
% \question[1] Выберите правильные ответы
% \begin{choices}
%     \choice    Симуляция, реализованная с помощью схемы PDES, всегда детерминистична
%     \choice Симуляция, реализованная с помощью схемы PDES, недетерминистична из-за возможности потери сообщений между потоками
%     \choice Симуляция, реализованная с помощью схемы PDES, недетерминистична из-за возможности блокировки отдельных потоков
%     \choice Симуляция, реализованная с помощью схемы PDES, недетерминистична из-за варьирующейся скорости работы отдельных потоков
% \end{choices}

% \question[1] Выберите правильные ответы
% \begin{choices}
%     \correctchoice Консервативные схемы PDES не допускают нарушения каузальности
%     \choice Консервативные схемы PDES допускают нарушения каузальности
%     \choice Консервативные схемы PDES допускают нарушения каузальности, но впоследствии их корректируют
%     \choice Оптимистичные схемы PDES не допускают нарушения каузальности
%     \choice Оптимистичные схемы PDES допускают нарушения каузальности
%     \correctchoice Оптимистичные схемы PDES допускают нарушения каузальности, но впоследствии их корректируют
% \end{choices}
    
% \question[1] Выберите правильные варианты
% \begin{choices}
%     \choice    Потактовые модели не могут быть написаны с помощью DES
%     \choice    Потактовые модели могут быть написаны с помощью DES, однако скорость работы будет низкой
%     \choice Потактовые модели не могут  быть написаны как  модели исполняющих устройств 
%     \correctchoice Потактовые модели могут быть сделанв исполняющими устройствами, однако такие модели будут негибкими и неудобными для модификации
%     \choice Потактовые модели могут быть написаны только с помощью концепции портов
% \end{choices}
%     
% \question[1] Выберите правильные варианты продолжения фразы: в модели, описанной на основе портов,
% \begin{choices}
%     \choice функциональные модули имеют различные задержки выполнения
%     \correctchoice функциональные модули не имеют определённой задержки выполнения
%     \choice функция портов является функцией тождественности, а задержка нулевая
%     \correctchoice функция портов является функцией тождественности, а задержка ненулевая
%     \choice функция портов не является функцией тождественности, а задержка нулевая
% \end{choices}
    

% \question[1] Что из нижеперечисленного может входить в трассу, используемую для симуляции?
% \begin{choices}
%     \correctchoice доступы во внешнюю память
%     \correctchoice внешние прерывания
%     \correctchoice временные метки
%     \choice состояние регистров
%     \choice дизассемблер текущих инструкций
% \end{choices}


\endgradingrange{morequestions}

\begingradingrange{chapterquestions}


% \question[1] Как расшифровается обозначение <<RTL-модель>> в контексте разработки аппаратуры?
% \begin{choices}
% \choice run-time library
% \correctchoice register transfer level
% \choice register-transistor logic
% \end{choices}


% \question[1] Какой из указанных ниже бенчмарков используется для оценки и сравнения эффективности работы систем виртуализации?
% \begin{choices}
% \choice SPECfp
% \choice SPECpower
% \choice SPECint
% \correctchoice SPECvirt
% \choice SPECjbb
% \end{choices}


% \question[1] Какие типы событий должны быть отражены в трассе работы приложения для того, чтобы она была полезна?
% \begin{choices}
% \correctchoice Только внешние события: доступы в память, к устройствам.
% \choice Только внутренние события: изменения регистров, 
% \choice И внутренние, и внешние события.
% \end{choices}

% \question[1] Как называется методика, призванная уменьшить объём данных трассы, требуемых для анализа работы приложения?
% \begin{choices}
% 	\choice Манипулирование.
% 	\choice Фильтрация.
% 	\choice Разогрев.
% 	\choice Интегрирование.
% 	\correctchoice Сэмплирование.
% \end{choices}

% \question[1] Выберите правильный порядок операций при обработке трассы.
% \begin{choices}
% 	\choice Перематывание -- измерение -- -- разогрев.
% 	\choice Разогрев -- перематывание  -- измерение.
% 	\correctchoice Перематывание -- разогрев -- измерение.
% \end{choices}


% \question[1] Сформулируйте соотношение для времени отклика системы.
% \begin{solution}[1cm]
% $R = N/X - Z$
% \end{solution}
% 
% \question[1] Сформулируйте соотношение для времени отклика системы.
% \begin{solution}[1cm]
% $R = N/X - Z$
% \end{solution}
% 
% \question[1] Какая из методик изучения опирается на генерацию случайных внешних воздействий?
% \begin{choices}
% 	\choice Изучение сетей обслуживания.
% 	\correctchoice Метод Монте-Карло.
% 	\choice Функциональная симуляция.
% 	\choice Потактовая симуляция.
% \end{choices}

% \question[1] Какие требования  выдвигаются генератору случайных чисел при их использовании в симуляторе?
% \begin{choices}
% 	\correctchoice Случайность и взаимная независимость.
% 	\choice Возможность изменять функцию распределения генерируемой последовательности.
% 	\correctchoice Высокая скорость генерации случайной последовательности.
% 	\choice Максимально возможная ширина выдаваемых чисел.
% 	\choice Криптографическая стойкость создаваемой последовательности.
% \end{choices}
% 
% \question[1] Выберите правильные свойства домена синхронизации в модели PDES
% \begin{choices}
% 	\choice Количество моделируемых устройств внутри одного домена фиксировано.
% 	\choice Не происходит взаимодействия устройств, находящихся в различных доменах.
% 	\choice Количество моделируемых устройств внутри одного домена ограничено.
% 	\correctchoice Характерная частота коммуникаций  между доменами превышает частоту коммуникаций внутри каждого.
% 	\choice Характерная частота коммуникаций  между доменами равна частоте коммуникаций внутри отдельного домена.
% \end{choices}
% 
% \question[1] Какую стратегию подразумевает концепция ленивого вычисления?
% \begin{choices}
% 	\choice Замена точного значения выражения приближённым, но получаемым за меньшее время.
% 	\correctchoice Запуск вычисления выражения происходит лишь при необходимости использовать его результата.
% 	\choice Выражение вычисляется сразу после доступности значений всех его входных слагаемых.
% 	\choice Значение подвыражения, используемого в нескольких других выражениях, сохраняется при первом вычислении и затем используется во всех них.
% \end{choices}
% 
% \question[1] Сколько бит информации получает процессор при первоначальном возникновении сигнала на линии прерывания?
% \begin{choices}
% 	\correctchoice 1 бит.
% 	\choice 8 бит.
% 	\choice Зависит от архитектуры.
% \end{choices}
% 
% \question[1] Какой интегральный тип языка Си наиболее удачно использовать для хранения состояния моделируемого регистра шириной 32 бита?
% \begin{choices}
% 	\choice \texttt{int}
% 	\choice \texttt{unsigned int}
% 	\correctchoice \texttt{uint32_t}
% 	\choice Зависит от хозяйской системы.
% \end{choices}

% \question[1] Выберите правильные окончания фразы: Карта памяти (memory map) \dots
% \begin{choices}
% 	\correctchoice использует цель по умолчанию, если обрабатываемый запрос не попадает ни в одно из устройств.
% 	\choice может содержать некторое устройство не более одного раза.
% 	\choice должна указывать на все присутствующие в гостевой системе устройства.
% 	\correctchoice может указывать не только на устройства, но и другие карты памяти.
% \end{choices}
% 

% \question[1] Данные могут попадать в кэш при следующих операциях:
% \begin{choices}
% 	\correctchoice чтение памяти (load)
% 	\correctchoice запись в память (store)
% 	\choice арифметические операции
% 	\choice операции с числами с плавающей запятой
% 	\correctchoice предвыборка данных (prefetch)
% 	\correctchoice загрузка инструкции (fetch)
% 	\choice инвалидация линии (invalidate)
% \end{choices}

% \question[1] Кэши необходимо симулировать даже в функциональной модели, если они используются для \dots
% \begin{choices}
% 	\correctchoice Создания транзакционной памяти.
% 	\choice Моделирования работы ЭВМ гарвардской архитектуры.
% 	\choice поддержания когерентности в SMP системах.
% \end{choices}

% \question[1] Какое утверждение наилучшим образом характеризует термин SystemC?
% \begin{choices}
% 	\choice Компилятор языка Си с дополнениями для моделирования систем.
% 	\choice Язык программирования, похожий на Си.
%     \choice Язык программирования, похожий на С++.
% 	\correctchoice Набор библиотек для С++.
% \end{choices}


% \question[1] Какое утверждение наилучшим образом характеризует термин TLM?
% \begin{choices}
% 	\choice Язык программирования, похожий на Си.
% 	\correctchoice Расширение стандарта SystemC.
%     \choice Язык программирования, похожий на С++.
%     \choice Среда исполнения моделей DES.
% \end{choices}
% 
% \question[1] Язык DML используется для разработки
% \begin{choices}
% 	\correctchoice Функциональных моделей.
%     \choice Потактовых моделей.
%     \choice Гибридных моделей.
% \end{choices}
% 
% \question[1] Язык DML используется для разработки
% \begin{choices}
% 	\correctchoice Неисполняющих моделей.
%     \choice Исполняющих моделей.
%     \choice Как исполняющих, так и неисполняющих моделей.
% \end{choices}
% 
% \question[1] Текущая реавлизация комилятора DMLC является \dots
% \begin{choices}
%     \choice Компилятором типа source-to-source с промежуточным языком С++.
%     \choice Компилятором, преобразующим исходный текст в байткод Java.
% 	\correctchoice Компилятором типа source-to-source с промежуточным языком Си.
%     \choice Классическим компилятором.
%     \choice Частичным интерпретатором.
% \end{choices}
% 
% \question[1] Какой способ наиболее удобен и надёжен для поддержания набора инструментов моделирования в синхронизированном состоянии при постоянном изменении входной спецификации процессора?
% \begin{choices}
% 	\correctchoice Генерация всех инструментов из единого описания.
%     \choice Тщательное сравнение всех инструментов после каждого изменения одного из них.
%     \choice Создание одного инструмента, поддерживающего максимальное количество функций.
% \end{choices}
% 
% \question[1] Закончите фразу: Языки разработки аппаратуры \dots
% \begin{choices}
% \choice не используются для начального моделирования устройств, так как могут быть преобразованы только в netlist.
% \correctchoice не используются для начального моделирования устройств, так как получаемые модели очень медленны.
% \choice не используются для начального моделирования устройств, так как могут содержать в себе синтезируюмую часть.
% \choice используются для начального моделирования устройств.
% \end{choices}
% 
% \question[1] Закончите фразу: Синтезируемое подмножество языков разработки аппаратуры \dots
% \begin{choices}
% \choice не может быть использовано для создания netlist и RTL-описаний.
% \choice используется только для отладки моделей.
% \correctchoice  используется для создания netlist и RTL-описаний.
% \end{choices}
% 
% \question[1] Назовите приём виртуализации, в котором гостевое приложение модифицируются таким образом, чтобы задействовать некоторую функциональность аппаратуры, присутствующую только внутри модели, но не на реальных системах?
% \begin{choices}
% \choice гиперсимуляция.
% \choice метавиртуализация.
% \correctchoice  паравиртуализация.
% \choice изоляция.
% \end{choices}


% \question[1] Для какой из перечисленных ниже операционных систем паравиртуализационные расширения сложно писать из-за закрытости исходного кода?
% \begin{choices}
% \correctchoice Microsoft Windows
% \choice GNU/Linux
% \choice FreeBSD
% \end{choices}
% 
% \question[1] Дайте определение термину <<проброс устройства>>.
% \begin{choices}
% \choice Передача устройства в эксклюзивное пользование нескольким гостям.
% \choice Передача устройства в эксклюзивное пользование хозяину.
% \correctchoice Передача устройства в эксклюзивное пользование единственному гостю.
% \end{choices}
% 
% \question[1] В чём состоят недостатки сырого формата дисков?
% \begin{choices}
% \choice невозможность случайного доступа к секторам диска
% \choice нерациональное расходование дискового пространства гостя.
% \correctchoice нерациональное расходование дискового пространства хозяина.
% \choice отсутствие публичной документации на формат.
% \end{choices}
% 
% \question[1] Для чего используются разностные файлы?
% \begin{choices}
% \correctchoice хранение изменений гостевого диска за время работы симуляции.
% \choice сжатие оригинального образа гостевого диска для того, чтобы он занимал меньше места.
% \choice прозрачное шифрование оригинального образа гостевого диска.
% \choice расширения размера гостевого диска в случае, когда старый полностью заполнен.
% \end{choices}

% \question[1] Перечислите все правильные виды сложностей, возникающих при разработке цифровых систем, успешно решаемые с помощью моделирования.
% \begin{choices}
% \correctchoice Необходимость знать характеристики новой технологии как можно раньше.
% \correctchoice Необходимость выявления ошибок проектирования на ранних стадиях.
% \choice Большое энергопотребление реальных образцов.
% \end{choices}

% \question[1] Почему симулятор не имеет права кэшировать регионы гостевой памяти, помеченные как отображённые на устройства?
% \begin{solution}[2cm]
% В отличие от простой памяти, хранящей данные без изменений, устройство может менять своё состояние и соответственно содержимое регионов памяти на каждом доступе к нему.
% \end{solution}


\endgradingrange{chapterquestions}

% ============================================================
\begingradingrange{bonuswork}
\bonusquestion \textit{Баллы за работу в течение семестра}
\endgradingrange{bonuswork}

% Some extra blank pages
\newpage
\phantom{Blank page}
\newpage
\phantom{Blank page}

\end{questions}
\end{document}

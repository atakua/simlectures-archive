% Compile with XeLaTeX only
%
%\documentclass[a4paper, addpoints, answers]{exam}
\documentclass[a4paper, addpoints]{exam}

\pointpoints{б.}{б.}
\bonuspointpoints{б.}{б.}

\def \variant {4} % muhaha

\usepackage{fontspec}
%\usepackage[utf8]{inputenc} % uncomment for latex2rtf 

\usepackage{xunicode} % some extra unicode support
\usepackage{xltxtra}

\usepackage{amsfonts}
\usepackage{amsmath}

\usepackage{csquotes}

\usepackage{polyglossia}
\defaultfontfeatures{Scale=MatchLowercase, Mapping=tex-text}

\newfontfamily\russianfont{TextBookC} % Гарнитура "Букварная"
\setromanfont{CMU Serif}
\setsansfont{CMU Sans Serif}
\setmonofont{CMU Typewriter Text}

\setdefaultlanguage[spelling=modern]{russian}
\setotherlanguage{english}
\newcommand{\todo}{{\color{red}TODO}\ }
\usepackage{graphicx}

\tolerance=9999 % let the text underfull be ugly as hell, nobody cares.
\emergencystretch=3cm

\usepackage{hyperref}
\hypersetup{colorlinks=true, linkcolor=black, filecolor=black, citecolor=black, urlcolor=black , pdfauthor=Grigory Rechistov, pdftitle=Контрольная работа по курсу <<Программное моделирование вычислительных систем>>}

\usepackage{footnpag}
\usepackage{indentfirst}
\usepackage{underscore}
\usepackage{url}

\usepackage{listings}
\lstset{basicstyle=\footnotesize\ttfamily, breaklines=true, keepspaces=true}

\usepackage{nameref}
\usepackage{amsthm}
\usepackage{enumitem} % continue enumeration 
\usepackage{subfigure}
\usepackage{mdwlist} % compact itemize lists environment

\setcounter{tocdepth}{2}

\renewcommand{\solutiontitle}{\noindent\textbf{Ответ:}\enspace}

\title{Программное моделирование вычислительных систем. Контрольная работа. Вариант \textnumero \variant}
\author{}
\newcommand{\testdate}{19 мая 2014 г.}
\date{\testdate}

\typeout{Copyright 2014 Grigory Rechistov.}
\begin{document}

\pagestyle{headandfoot}
\runningheadrule
\firstpageheader{Вариант \textnumero \variant}{Кафедра <<Микропроцессорные технологии>>}{\testdate}
\runningheader{Вариант \textnumero \variant}{Кафедра <<Микропроцессорные технологии>>}{\testdate}
\coverfooter{}{Стр. \thepage\ из \numpages}{}
\firstpagefooter{}{Стр. \thepage\ из \numpages}{}
\runningfooter{}{Стр. \thepage\ из \numpages}{}

\maketitle\thispagestyle{headandfoot} % a hack to have proper footers and headers on the first page as well

\medskip
\makebox[0.9\textwidth]{Ф.И.О.\enspace\hrulefill}
\medskip

\makebox[0.9\textwidth]{Группа\enspace\hrulefill}
\medskip

\makebox[0.9\textwidth]{e-mail\enspace\hrulefill}

% Grade tables
\hqword{Вопрос}
\hpword{Баллов}
\hsword{Результат}
\htword{Сумма}
\vqword{Вопрос}
\vpword{Баллов}
\vsword{Результат}
\vtword{Сумма}
\cellwidth{0.5em}
% \gradetablestretch{1.8}
\begin{center}
\tiny
\gradetable[h][questions]
% \partialgradetable{multiplechoice}[h][questions]\\
% \partialgradetable{fullquestions}[h][questions]\\
% \partialgradetable{bonuswork}[h][questions]\\
\end{center}

\medskip

\begin{questions}

% Introduce several grading range so that grading tables are short enough to fit on the page

% \begingradingrange{multiplechoice} 

% lecture01-why-simulation
\question[2] Определите понятие <<потактовая симуляция>>  в контексте моделирования компьютеров.
\begin{solution}[2cm]
\end{solution}

% lecture02-common-questions
\question[2] Определите понятие <<Shift left>> в применении к разработке программно-аппаратных систем (железа и ПО).
\begin{solution}[2cm]
\end{solution}

% lecture03-cpu-interpretation
% lecture03pt2-cpu-interpretation
\question[2] Объясните принцип работы сцепленного (threaded) интерпретатора. Почему он оказывается быстрее обычного переключаемого?
\begin{solution}[2cm]
\end{solution}

% lecture04-bt-pt1
% lecture04pt2-bt
\question[2] Опишите существующие приёмы для обеспечения работы симуляции с прямым исполнением гостевого кода в условиях присутствия в нём небезопасных инструкций?
\begin{solution}[2cm]
\end{solution}

% lecture05-des
\question[2] Опишите, какие условия на изменения симулируемого времени должны выполняться в системе, состоящей как из исполняющих, так и неисполняющих моделей устройств?
\begin{solution}[2cm]
\end{solution}

% lecture06-pdes1
% lecture07-pdes2
% lecture07-pdes3
\question[2] Каким образом в оптимистичных моделях PDES обеспечивается корректность причинно-следственного порядка моделируемых событий?
\begin{solution}[2cm]
\end{solution}

\question[2] Опишите, для каких нужд в архитектурах ЦПУ вводятся атомарные инструкции. Каким образом они могут эмулироваться?
\begin{solution}[2cm]
\end{solution}

% lecture08-cycle-precise
\question[1] Выберите правильные варианты ответов: для потактовой модели на основе портов
\begin{choices}
    \correctchoice при передаче данных порты сохраняют бит валидности данных,
    \choice при передаче данных порты не сохраняют бит валидности данных,
    \choice при передаче данных порты не сохраняют бит валидности данных, только если он снят,
    \choice при передаче данных порты не сохраняют бит валидности данных, только если он поднят.
\end{choices}

% lecture09-paravirt
\question[2] Опишите преимущества и недостатки использования паравиртуальных устройств внутри симуляции. Какие классы устройств подвергаются паравиртуализации в первую очередь?
\begin{solution}[2cm]
\end{solution}

% lecture13-efficient-virt
\question[2] Дайте определение понятия <<служебная инструкция>> в терминах принципа Голдберга и Попека.
\begin{solution}[2cm]
Инструкция, изменяющая системные ресурсы даже из пользовательского режима, или инструкция, семантика которой зависит от режима процессора.
\end{solution}

% lecture12-languages
\question[1] Какое утверждение наилучшим образом характеризует термин TLM?
\begin{choices}
    \correctchoice Набор библиотек для С++.
    \choice Компилятор языка Си с дополнениями для моделирования систем.
    \choice Компилятор языка С++ с дополнениями для моделирования систем.
    \choice Язык программирования, похожий на Си.
\end{choices}

% ============================================================
\begingradingrange{bonuswork}
\bonusquestion \textit{Баллы за работу в течение семестра}
\endgradingrange{bonuswork}

% Some extra blank pages
\newpage
\phantom{Blank page}

\end{questions}
\end{document}

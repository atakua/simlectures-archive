% Compile with XeLaTeX only
%
\documentclass[a4paper, addpoints, answers]{exam}
%\documentclass[a4paper, addpoints]{exam}

\pointpoints{б.}{б.}
\bonuspointpoints{б.}{б.}

\def \variant {2}

\usepackage{fontspec}
%\usepackage[utf8]{inputenc} % uncomment for latex2rtf 

\usepackage{xunicode} % some extra unicode support
\usepackage{xltxtra}

\usepackage{amsfonts}
\usepackage{amsmath}

\usepackage{csquotes}

\usepackage{polyglossia}
\defaultfontfeatures{Scale=MatchLowercase, Mapping=tex-text}

\newfontfamily\russianfont{CMU Serif}
\setromanfont{CMU Serif}
\setsansfont{CMU Sans Serif}
\setmonofont{CMU Typewriter Text}

\setdefaultlanguage[spelling=modern]{russian}
\setotherlanguage{english}
\newcommand{\todo}{{\color{red}TODO}\ }
\usepackage{graphicx}

\tolerance=9999 % let the text underfull be ugly as hell, nobody cares.
\emergencystretch=3cm

\usepackage{hyperref}
\hypersetup{colorlinks=true, linkcolor=black, filecolor=black, citecolor=black, urlcolor=black , pdfauthor=Grigory Rechistov, pdftitle=Контрольная по курсу <<Основы программного моделирования ЭВМ>>}

\usepackage{footnpag}
\usepackage{indentfirst}
\usepackage{underscore}
\usepackage{url}

\usepackage{listings}
\lstset{basicstyle=\footnotesize\ttfamily, breaklines=true, keepspaces=true}

\usepackage{nameref}
\usepackage{amsthm}
\usepackage{enumitem} % continue enumeration 
\usepackage{subfigure}
\usepackage{mdwlist} % compact itemize lists environment

\setcounter{tocdepth}{2}

\renewcommand{\solutiontitle}{\noindent\textbf{Ответ:}\enspace}

\title{Основы программного моделирования ЭВМ. Контрольная работа. Вариант \textnumero \variant}
\author{}
\newcommand{\testdate}{\today}
\date{\testdate}

\typeout{Copyright 2012 Grigory Rechistov and the contributors.}
\begin{document}

\pagestyle{headandfoot}
\runningheadrule
\firstpageheader{Вариант \textnumero \variant}{Кафедра <<Микропроцессорные технологии>>}{\testdate}
\runningheader{Вариант \textnumero \variant}{Кафедра <<Микропроцессорные технологии>>}{\testdate}
\coverfooter{}{Стр. \thepage\ из \numpages}{}
\firstpagefooter{}{Стр. \thepage\ из \numpages}{}
\runningfooter{}{Стр. \thepage\ из \numpages}{}

\maketitle\thispagestyle{headandfoot} % a hack to have proper footers and headers on the first page as well

\medskip
\makebox[0.9\textwidth]{Ф.И.О.\enspace\hrulefill}
\medskip

\makebox[0.9\textwidth]{Группа\enspace\hrulefill}
\medskip

\makebox[0.9\textwidth]{e-mail\enspace\hrulefill}

% Grade tables
\hqword{Вопрос}
\hpword{Баллов}
\hsword{Результат}
\htword{Сумма}
\vqword{Вопрос}
\vpword{Баллов}
\vsword{Результат}
\vtword{Сумма}
\cellwidth{0.5em}
\gradetablestretch{1.8}
\begin{center}
\tiny
% \gradetable[h][questions]
\partialgradetable{multiplechoice}[h][questions]\\
\partialgradetable{fullquestions}[h][questions]\\
\partialgradetable{bonuswork}[h][questions]\\
\end{center}

\medskip

% \begin{center}
% \fbox{\fbox{\parbox{\textwidth}{\centering
% В заданиях выделите правильные варианты ответов или дайте развёрнутый ответ.
% }}}
% \end{center}

\begin{questions}

lecture01-why-simulation
% lecture02-common-questions
% lecture03-cpu-interpretation
% lecture03pt2-cpu-interpretation
% lecture04-bt-pt1
% lecture04pt2-bt
% lecture05-des
% lecture06-pdes1
% lecture07-pdes2
% lecture07-pdes3
% lecture08-cycle-precise
% lecture09-paravirt
% lecture13-efficient-virt
% lang

\begingradingrange{multiplechoice} 

\question[1] Какие из указанных ниже компонентов обязательны для реализации интерпретатора?
\begin{choices}
    \correctchoice декодер
    \choice дизассемблер
    \choice енкодер
    \correctchoice блоки реализации семантики отдельных инструкций
    \choice кэш декодированных инструкций
\end{choices}

\question[1] Выберите правильный вариант окончания фразы: в моделях DES новые события могут быть добавлены
\begin{choices}
    \choice только к голове очереди событий
    \choice только к хвосту очереди событий
    \correctchoice в любую позицию в очереди событий
\end{choices}

%\question[1] Выберите все необходимые условия корректности применения гиперсимуляции процессора.
%\begin{choices}
%    \correctchoice Нет обращений к внешней памяти
%    \choice  Нет обращений к внешним устройствам
%    \choice Только один процессор в системе
%    \correctchoice Состояние внешних устройств не меняется
%    \choice состояние процессора не меняется
%\end{choices}

\question[1] Какие из типов схем PDES позволяют добиться детерминизма симуляции?
\begin{choices}
    \correctchoice Барьерная (с доменами синхронизации)
    \choice Консервативная (с пересылкой пустых сообщений)
    \choice Консервативная (с блокировкой отправителя)
    \choice Оптимистичная
    \choice Наивная
\end{choices}


%\question[1] Выберите правильные варианты продолжения фразы: процесс исполнения потактовой модели на основе портов 
%\begin{choices}
%    \choice всегда содержит две фазы, порядок которых не фиксированный
%    \choice всегда содержит одну фазу, в течение которой работают все субъединицы
%    \choice может содержать более двух чередующихся фаз
%    \correctchoice всегда содержит две фазы, которые обязаны чередоваться
%\end{choices}

%\question[1] Выберите правильные варианты завершения фразы: в потактовых моделях, созданных на основе концепции портов
%\begin{choices}
%	\choice при передаче данных порты не сохраняют бит валидности данных
%    \choice при передаче данных порты не сохраняют бит валидности данных, только если он снят
%    \choice при передаче данных порты не сохраняют бит валидности данных, только если он поднят
%    \correctchoice при передаче данных порты сохраняют бит валидности данных
%\end{choices}

\question[1] Выберите правильный вариант окончания фразы: сцепленный интерпретатор работает быстрее переключаемого (switched), так как
\begin{choices}
    \correctchoice удачно использует предсказатель переходов хозяйского процессора
    \choice кэширует недавно исполненные инструкции
    \choice транслирует код в промежуточное представление
    \choice не требует обработки исключений
\end{choices}

\question[1] Для какого класса схем PDES события в симулируемом прошлом не могут быть удалены сразу после их обработки, а требуют специальной процедуры <<fossil collection>>?
\begin{choices}
    \choice Барьерная (с доменами синхронизации)
    \choice Консервативная (с пересылкой пустых сообщений)
    \choice Консервативная (с блокировкой отправителя)
    \correctchoice Оптимистичная
\end{choices}


\question[1] Выберите правильный вариант окончания фразы: в моделях DES события могут обрабатываться, если они находятся
\begin{choices}
    \choice только в голове очереди событий (самые поздние)
    \correctchoice только в хвосте очереди событий (самые ранние)
    \choice на любой позиции в очереди событий
\end{choices}

\question[1] Какой байт будет расположен по младшему адресу в памяти на Little Endian системе при записи числа \texttt{0xaabbccdd}?
\begin{solution}[1cm]
0xdd
\end{solution}

%\question[1] Выберите правильные варианты
%\begin{choices}
%    \choice    Темпы роста скорости оперативной памяти и процессоров одинаковы с 80-х годов 20 века.
%    \choice Темп роста скорости оперативной памяти опережает темпы роста скорости работы процессоров
%    \correctchoice Темп роста скорости процессоров опережает темпы роста скорости оперативной памяти 
%\end{choices}

%\question[1] Выберите правильные варианты: линия данных с фиксированным адресом
%\begin{choices}
%    \choice всегда попадает в одну и ту же ячейку кэша
%    \correctchoice всегда попадает в один и тот же сет кэша
%    \choice может быть сохранена в любой ячейке кэша
%\end{choices}

%\question[1] Какие сценарии представляют наибольшую сложность для метода симуляции с помощью трасс?
%\begin{choices}
%\choice Гостевое приложение с закрытым исходным кодом.
%\choice Изучение производительности приложений.
%\correctchoice Многопоточная гостевая система.
%\end{choices}

\question[1] Какие из данных инструкций архитектуры IA-32, скорее всего, не могут быть использованы в качестве волшебных?
\begin{choices}
	\choice \texttt{CPUID} --- идентификация процессора.
	\correctchoice \texttt{INT} --- привилегированное программное прерывание.
	\choice \texttt{NOP} --- пустая операция.
	\correctchoice \texttt{JMP} --- безусловный переход.
\end{choices}


\question[1] Известно, что архитектура A имеет модель консистентности памяти, менее строгую, чем архитектура B, но более строгую, чем архитектура С. На каких хозяйских архитектурах симуляция памяти гостевой А не будет требовать использования явных барьеров?
\begin{choices}
    \correctchoice На A.
    \correctchoice На B.
    \choice На С.
    \choice На B и C.
    \choice Явная синхронизация потребуется во всех случаях.
\end{choices}


\endgradingrange{multiplechoice}

\begingradingrange{fullquestions}

\question[3] Дайте определение понятия <<полноплатформенный симулятор>>
\begin{solution}[2cm]
Симулятор,  способный запускать операционные системы и потому содержащий модели периферийных устройств.
\end{solution}

% \question[1] Дайте определение понятия <<потактовый симулятор>>
% \begin{solution}[2cm]
% Модель, обеспечивающая корректное выполнение алгоритмов отдельных инструкций и при этом высчитывающая задержки, возникающие при их исполнении.
% \end{solution}

\question[3] Опишите, что происходит на стадии \textbf{Fetch} работы процессора.
\begin{solution}[1cm]
Чтение из памяти машинного кода, соответствующего текущей инструкции.
\end{solution}

% \question[1] Опишите, что происходит на стадии \textbf{Writeback} работы процессора. Для каких инструкций эта стадия будет опущена?
% \begin{solution}[2cm]
% Запись результатов исполнения инструкции в память. Если результат должен быть сохранён в регистре, то фаза опускается.
% \end{solution}

\question[3] Какой вид программ обычно исполняется в непривилегированном режиме процессора?
\begin{solution}[1cm]
Пользовательские приложения.
\end{solution}

\question[3] Дайте определение понятия <<машинное слово>>.
\begin{solution}[1cm]
Максимальный объём данных, который процессор данной архитектуры способен обработать за одну инструкцию. 
\end{solution}

\question[3] Почему самый простой вид декодера машинных инструкций --- однотабличный --- не пользуется большой популярностью?
\begin{solution}[1cm]
    Размер таблицы растёт экспоненциально от длины опкода. Для архитектур, использующих больше 16 бит для кодирования инструкций, такая таблица становится несообразно огромной.
\end{solution}

\question[3] Дайте определение гипервизора первого типа.
\begin{solution}[2cm]
Гипервизоры первого типа (автономные гипервизоры) работают прямо на хозяйской аппаратуре, т.е. не требуют для своей работы операционной системы, беря её функции на себя и являясь привилегированными приложениями.
\end{solution}

%\question[3] Дайте определение величине MIPS, используемой для измерения скорости  программ.
%\begin{solution}[1cm]
%Количество миллионов инструкций, исполняющихся за одну секунду.
%\end{solution}

\question[3] Возможна ли указанная на рисунке ситуация в модели DES? Почему?
\begin{center}
\includegraphics[width=0.7\textwidth]{wrong-des-crop}
\end{center}
\begin{solution}[1cm]
Ситуация невозможна. Событие не может быть левее счётчика текущего виртуального времени.
\end{solution}

\question[3] Перечислите отличия двоичной трансляции от компиляции с языков высокого уровня, мешаюшие применению классических оптимизаций последних.
\begin{solution}[3cm]
В входном языке ДТ отсутствуют имена переменных, границы блоков алгоритмов, нет разделения между кодом и данными.
\end{solution}

%\question[3] Сформулируйте закон баланса потока
%\begin{solution}[1cm]
%Скорость прибытия клиентов равна пропускной способности системы: $\lambda = X$.
%\end{solution}

\question[3] Дайте определение понятия <<исполняющее устройство>>
\begin{solution}[1cm]
Модель, изменение состояния которой контролируется ею самой, изнутри.
\end{solution}

\question[3] Как могут проявиться недостатки слишком большой квоты при симуляции многопроцессорной системы в составе DES?
\begin{solution}[2cm]
    Симуляция будет некорректной --- будут происходить таймауты взаимодействия моделируемых процессоров.
\end{solution}

\question[1] Почему симулятор не имеет права кэшировать регионы гостевой памяти, помеченные как отображённые на устройства?
\begin{solution}[2cm]
В отличие от простой памяти, хранящей данные без изменений, устройство может менять своё состояние и соответственно содержимое регионов памяти на каждом доступе к нему.
\end{solution}


\endgradingrange{fullquestions} 


% \question[1] Перечислите все правильные виды сложностей, возникающих при разработке цифровых систем, успешно решаемые с помощью моделирования.
% \begin{choices}
% \correctchoice Большое число составляющих систему устройств со сложными взаимосвязями.
% \choice Сложность получения лицензий на новое оборудование.
% \correctchoice Обеспечение поддержки аппаратуры программными средствами разработки.
% \end{choices}

\begingradingrange{bonuswork}
\bonusquestion \textit{Баллы за работу в течение семестра}
\endgradingrange{bonuswork}

% Some extra blank pages
\newpage
\phantom{Blank page}
%\newpage
%\phantom{Blank page}



\end{questions}

\end{document}

% Compile with XeLaTeX only
%
%\documentclass[a4paper, addpoints, answers]{exam}
\documentclass[a4paper, addpoints]{exam}

\pointpoints{б.}{б.}
\bonuspointpoints{б.}{б.}

\def \variant {152} % muhaha

\usepackage{fontspec}
%\usepackage[utf8]{inputenc} % uncomment for latex2rtf 

\usepackage{xunicode} % some extra unicode support
\usepackage{xltxtra}

\usepackage{amsfonts}
\usepackage{amsmath}

\usepackage{csquotes}

\usepackage{polyglossia}
\defaultfontfeatures{Scale=MatchLowercase, Mapping=tex-text}

\newfontfamily\russianfont{TextBookC} % Гарнитура "Букварная"
\setromanfont{CMU Serif}
\setsansfont{CMU Sans Serif}
\setmonofont{CMU Typewriter Text}

\setdefaultlanguage[spelling=modern]{russian}
\setotherlanguage{english}
\newcommand{\todo}{{\color{red}TODO}\ }
\usepackage{graphicx}

\tolerance=9999 % let the text underfull be ugly as hell, nobody cares.
\emergencystretch=3cm

\usepackage{hyperref}
\hypersetup{colorlinks=true, linkcolor=black, filecolor=black, citecolor=black, urlcolor=black , pdfauthor=Grigory Rechistov, pdftitle=Контрольная работа по курсу <<Программное моделирование вычислительных систем>>}

\usepackage{footnpag}
\usepackage{indentfirst}
\usepackage{underscore}
\usepackage{url}

\usepackage{listings}
\lstset{basicstyle=\footnotesize\ttfamily, breaklines=true, keepspaces=true}

\usepackage{nameref}
\usepackage{amsthm}
\usepackage{enumitem} % continue enumeration 
\usepackage{subfigure}
\usepackage{mdwlist} % compact itemize lists environment

\setcounter{tocdepth}{2}

\renewcommand{\solutiontitle}{\noindent\textbf{Ответ:}\enspace}

\title{Программное моделирование вычислительных систем. Контрольная работа. Вариант \textnumero \variant}
\author{}
\newcommand{\testdate}{25 мая 2015 г.}
\date{\testdate}

\typeout{Copyright 2015 Grigory Rechistov.}
\begin{document}

\pagestyle{headandfoot}
\runningheadrule
\firstpageheader{Вариант \textnumero \variant}{Кафедра <<Микропроцессорные технологии>>}{\testdate}
\runningheader{Вариант \textnumero \variant}{Кафедра <<Микропроцессорные технологии>>}{\testdate}
\coverfooter{}{Стр. \thepage\ из \numpages}{}
\firstpagefooter{}{Стр. \thepage\ из \numpages}{}
\runningfooter{}{Стр. \thepage\ из \numpages}{}

\maketitle\thispagestyle{headandfoot} % a hack to have proper footers and headers on the first page as well

\medskip
\makebox[0.9\textwidth]{Ф.И.О.\enspace\hrulefill}
\medskip

\makebox[0.9\textwidth]{Группа\enspace\hrulefill}
\medskip

\makebox[0.9\textwidth]{e-mail\enspace\hrulefill}

% Grade tables
\hqword{Вопрос}
\hpword{Баллов}
\hsword{Результат}
\htword{Сумма}
\vqword{Вопрос}
\vpword{Баллов}
\vsword{Результат}
\vtword{Сумма}
\cellwidth{0.5em}
% \gradetablestretch{1.8}
\begin{center}
\tiny
\gradetable[h][questions]
% \partialgradetable{multiplechoice}[h][questions]\\
% \partialgradetable{fullquestions}[h][questions]\\
% \partialgradetable{bonuswork}[h][questions]\\
\end{center}

\medskip

\begin{questions}

% Introduce several grading range so that grading tables are short enough to fit on the page

% \begingradingrange{multiplechoice} 

% lecture01-why-simulation
% \question[2] Определите понятие <<виртуализация>>  в контексте моделирования компьютеров.
% \begin{solution}[2cm]
% \end{solution}
\question[2] Определите понятие <<гостевая система>> в контексте моделирования компьютеров.
\begin{solution}[2cm]
\end{solution}

% % lecture02-common-questions
% \question[2] Определите понятие <<Shift left>> в применении к разработке программно-аппаратных систем (железа и ПО).
% \begin{solution}[2cm]
% \end{solution}
% lecture02-common-questions
\question[2] Перечислите способы измерения скорости работы различных типов программных моделей.
\begin{solution}[2cm]
\end{solution}

% lecture03-cpu-interpretation
% \question[2] Объясните принцип работы сцепленного (threaded) интерпретатора. Почему он оказывается быстрее обычного переключаемого?
% \begin{solution}[2cm]
% \end{solution}
\question[2] Определите понятие <<декодирование>> в контексте создания программного симулятора ЦПУ (интерпретатора).
\begin{solution}[2cm]
\end{solution}

\question[2] Определите понятие <<асинхронное прерывание>>. Приведите пример ситуации, когда возникает это событие.
\begin{solution}[2cm]
\end{solution}

% 04-arch-state
\question[2] Опишите механизм ленивого вычисления флагов при симуляции арифметических инструкций, изменяющих флаги. Почему ленивое вычисление повышает скорость симуляции?
\begin{solution}[2cm]
\end{solution}

\question[2] Опишите, как работает архитектурный механизм Memory Mapped Input/Output (MMIO) для доступа к периферийным устройствам.
\begin{solution}[2cm]
\end{solution}

% lecture04-bt-pt1
% lecture04pt2-bt
\question[2] Опишите существующие способы обеспечения симуляции с прямым исполнением гостевого кода в условиях присутствия в нём небезопасных инструкций.
\begin{solution}[2cm]
\end{solution}

\question[2] Определите, что входит в понятие <<обнаружение кода>> при двоичной трансляции.
\begin{solution}[2cm]
\end{solution}


% lecture05-des
\question[2] Опишите, чем опасно использование излишне большого значения квоты при симуляции нескольких функциональных моделей ЦПУ.
\begin{solution}[2cm]
\end{solution}


\question[2] В каких условиях программный симулятор, построенный на основе принципов моделирования дискретных событий (DES), \emph{не будет} детерминистичным?
\begin{solution}[2cm]
\end{solution}

% lecture06-pdes1
% lecture07-pdes2
% lecture07-pdes3
\question[2] Каким образом в оптимистичных моделях PDES обеспечивается корректность причинно-следственного порядка моделируемых событий?
\begin{solution}[2cm]
\end{solution}

\question[2] Опишите, для каких нужд в архитектурах ЦПУ вводятся атомарные инструкции. Каким образом они могут эмулироваться?
\begin{solution}[2cm]
\end{solution}

% lecture08-cycle-precise
\question[1] Выберите правильные варианты ответов: для потактовой модели на основе портов
\begin{choices}
    \correctchoice при передаче данных порты сохраняют бит валидности данных,
    \choice при передаче данных порты не сохраняют бит валидности данных,
    \choice при передаче данных порты не сохраняют бит валидности данных, только если он снят,
    \choice при передаче данных порты не сохраняют бит валидности данных, только если он поднят.
\end{choices}

% lecture13-efficient-virt
\question[2] Определите свойство <<эквивалентность>>, требуемое для построения эффективного монитора виртуальных машин.
\begin{solution}[2cm]
\end{solution}

\question[2] Дайте определение понятия <<служебная инструкция>> в терминах принципа Голдберга и Попека.
\begin{solution}[2cm]
Инструкция, изменяющая системные ресурсы даже из пользовательского режима, или инструкция, семантика которой зависит от режима процессора.
\end{solution}


% ============================================================
\begingradingrange{bonuswork}
\bonusquestion \textit{Баллы за работу в течение семестра}
\endgradingrange{bonuswork}

% Some extra blank pages
\newpage
\phantom{Blank page}

\end{questions}
\end{document}

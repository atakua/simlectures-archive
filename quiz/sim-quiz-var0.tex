% Compile with XeLaTeX only
%
\documentclass[a4paper, 12pt, addpoints, answers]{exam}
%\documentclass[a4paper, 12pt, addpoints]{exam}

\pointpoints{б.}{б.}
\bonuspointpoints{б.}{б.}

\def \variant {1}

\usepackage{fontspec}
%\usepackage[utf8]{inputenc} % uncomment for latex2rtf 

\usepackage{xunicode} % some extra unicode support
\usepackage{xltxtra}

\usepackage{amsfonts}
\usepackage{amsmath}

\usepackage{csquotes}

\usepackage{polyglossia}
\defaultfontfeatures{Scale=MatchLowercase, Mapping=tex-text}

\newfontfamily\russianfont{TextBookC} % Гарнитура "Букварная"
\setromanfont{CMU Serif}
\setsansfont{CMU Sans Serif}
\setmonofont{CMU Typewriter Text}

\setdefaultlanguage[spelling=modern]{russian}
\setotherlanguage{english}
\newcommand{\todo}{{\color{red}TODO}\ }
\usepackage{graphicx}

\tolerance=9999 % let the text underfull be ugly as hell, nobody cares.
\emergencystretch=3cm

\usepackage{hyperref}
\hypersetup{colorlinks=true, linkcolor=black, filecolor=black, citecolor=black, urlcolor=black , pdfauthor=Grigory Rechistov <grigory.rechistov@phystech.edu>, pdftitle=Контрольная по курсу «Основы программного моделирования»}

\usepackage{footnpag}
\usepackage{indentfirst}
\usepackage{underscore}
\usepackage{url}

\usepackage{listings}
\lstset{basicstyle=\footnotesize\ttfamily, breaklines=true, keepspaces=true}

\usepackage{nameref}
\usepackage{amsthm}
\usepackage{enumitem} % continue enumeration 
\usepackage{subfigure}
\usepackage{mdwlist} % compact itemize lists environment

\setcounter{tocdepth}{2}

\renewcommand{\solutiontitle}{\noindent\textbf{Ответ:}\enspace}

\title{Основы программного моделирования. Контрольная работа. Вариант \textnumero \variant}
\author{}
% \newcommand{\testdate}{\today}
\newcommand{\testdate}{14 апреля 2014 г.}
\date{\testdate}

\typeout{Copyright 2014 Grigory Rechistov and the contributors.}
\begin{document}

\pagestyle{headandfoot}
\runningheadrule
\firstpageheader{Вариант \textnumero \variant}{\url{http://ilab.mipt.ru}}{\testdate}
\runningheader{Вариант \textnumero \variant}{\url{http://ilab.mipt.ru}}{\testdate}

% \firstpageheader{Вариант \textnumero \variant}{Кафедра «Микропроцессорные технологии»}{\testdate}
% \runningheader{Вариант \textnumero \variant}{Кафедра «Микропроцессорные технологии»}{\testdate}

\coverfooter{}{Стр. \thepage\ из \numpages}{}
\firstpagefooter{}{Стр. \thepage\ из \numpages}{}
\runningfooter{}{Стр. \thepage\ из \numpages}{}

\maketitle\thispagestyle{headandfoot} % a hack to have proper footers and headers on the first page as well

\medskip
\makebox[0.9\textwidth]{Ф.И.О.\enspace\hrulefill}
\medskip

\makebox[0.9\textwidth]{Группа\enspace\hrulefill}
\medskip

\makebox[0.9\textwidth]{e-mail\enspace\hrulefill}

% Grade tables
\hqword{Вопрос}
\hpword{Баллов}
\hsword{Результат}
\htword{Сумма}
\vqword{Вопрос}
\vpword{Баллов}
\vsword{Результат}
\vtword{Сумма}
\cellwidth{0.5em}
\gradetablestretch{1.8}
\begin{center}
\tiny
%\gradetable[h][questions]
\partialgradetable{multiplechoice}[h][questions]\\
\partialgradetable{fullquestions}[h][questions]\\
\partialgradetable{bonuswork}[h][questions]\\
\end{center}

\medskip

% Introduce several grading range so that grading tables are short enough to fit on the page

\begingradingrange{multiplechoice} 

\begin{questions}
\question[1] Выберите правильные варианты продолжения фразы: использование кэшей при работе приложения целесообразно, если
\begin{choices}
    \choice программа не обращается в оперативную память,
    \correctchoice    программа показывает временную локальность доступов,
    \choice программа работает с очень большим объёмом данных,
    \choice программа работает с объёмом данных, меньшим ёмкости кэша,
    \correctchoice программа показывает пространственную локальность доступов.
\end{choices}

\question[1] Выберите правильные варианты окончания: линия данных с фикированным адресом
\begin{choices}
    \correctchoice всегда попадает в один и тот же сет,
    \choice всегда попадает в одну и ту же ячейку кэша,
    \choice может быть сохранён в любой ячейке кэша.
\end{choices}

\question[1] Какое утверждение наилучшим образом характеризует термин SystemC?
\begin{choices}
    \choice Компилятор языка Си с дополнениями для моделирования систем.
    \choice Язык программирования, похожий на Си.
    \choice Язык программирования, похожий на С++.
    \correctchoice Набор библиотек для С++.
\end{choices}

\question[1] Какой способ наиболее удобен и надёжен для поддержания набора инструментов моделирования в синхронизированном состоянии при постоянном изменении входной спецификации процессора?
\begin{choices}
    \choice Тщательное сравнение всех инструментов после каждого изменения одного из них.
    \choice Создание одного инструмента, поддерживающего максимальное количество функций.
    \correctchoice Генерация всех инструментов из единого описания.
\end{choices}

\question[1] Какие сценарии представляют наибольшую сложность для метода симуляции с помощью трасс:
\begin{choices}
\correctchoice многопоточная гостевая система,
\choice гостевое приложение с закрытым исходным кодом,
\choice изучение производительности приложений?
\end{choices}

\question[1] Какие из типов схем PDES позволяют добиться детерминизма симуляции?
\begin{choices}
    \choice Наивная.
    \choice Консервативная.
    \choice Оптимистичная.
    \correctchoice Барьерная (с доменами синхронизации).
\end{choices}

\question[1] В чём состоят недостатки сырого формата дисков?
\begin{choices}
\choice невозможность случайного доступа к секторам диска,
\correctchoice нерациональное расходование дискового пространства хозяина,
\choice нерациональное расходование дискового пространства гостя,
\choice отсутствие публичной документации на формат.
\end{choices}


\endgradingrange{multiplechoice}

\begingradingrange{fullquestions}

\question[3] Моделирование какого типа общего аппаратного ресурса многопроцессорных гостей вызывает наибольшие сложности при создании параллельных симуляторов?
\begin{solution}[2cm]
 Общей оперативной памяти. Типичные приложения обращаются к ней достаточно часто, чтобы вызывать необходимость в синхронизации таких доступов при моделировании.
\end{solution}

\question[3] Определите понятие «атомарная инструкция».
\begin{solution}[2cm]
Инструкция, при исполнении которой гарантируется, что чтение и/или модификация региона общей памяти, используемого в ней, не будет пересекаться с доступом к ней из других потоков.
\end{solution}

\question[3] Опишите условия на сценарий симуляции, в котором оптимистичная схема PDES покажет наилучшую производительность.
\begin{solution}[2cm]
1) Редкие нарушения каузальности; 2) Низкая цена откатов в случаях возникновения нарушений.
\end{solution}

\question[3] Какие абстракции, необходимые в процессе проектирования аппаратуры, как правило, отсутствуют в языках программирования общего назначения?
\begin{solution}[2cm]
Сигналы нескольких логических уровней и типов (1, 0, Z, X), шины, операции над отдельными битами, транзакции.
\end{solution}


\endgradingrange{fullquestions} 


% ============================================================
\begingradingrange{bonuswork}
\bonusquestion \textit{Баллы за работу в течение семестра}
\endgradingrange{bonuswork}

% Some extra blank pages
\newpage
\phantom{Blank page}
%\newpage
%\phantom{Blank page}

\end{questions}
\end{document}

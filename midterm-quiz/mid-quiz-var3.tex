% Compile with XeLaTeX only
%
%\documentclass[a4paper, addpoints, answers]{exam}
\documentclass[a4paper, addpoints]{exam}

\pointpoints{б.}{б.}
\bonuspointpoints{б.}{б.}

\def \variant {3} % muhaha

\usepackage{fontspec}
%\usepackage[utf8]{inputenc} % uncomment for latex2rtf 

\usepackage{xunicode} % some extra unicode support
\usepackage{xltxtra}

\usepackage{amsfonts}
\usepackage{amsmath}

\usepackage{csquotes}

\usepackage{polyglossia}
\defaultfontfeatures{Scale=MatchLowercase, Mapping=tex-text}

\newfontfamily\russianfont{TextBookC} % Гарнитура "Букварная"
\setromanfont{CMU Serif}
\setsansfont{CMU Sans Serif}
\setmonofont{CMU Typewriter Text}

\setdefaultlanguage[spelling=modern]{russian}
\setotherlanguage{english}
\newcommand{\todo}{{\color{red}TODO}\ }
\usepackage{graphicx}

\tolerance=9999 % let the text underfull be ugly as hell, nobody cares.
\emergencystretch=3cm

\usepackage{hyperref}
\hypersetup{colorlinks=true, linkcolor=black, filecolor=black, citecolor=black, urlcolor=black , pdfauthor=Grigory Rechistov, pdftitle=Контрольная работа по курсу <<Программное моделирование вычислительных систем>>}

\usepackage{footnpag}
\usepackage{indentfirst}
\usepackage{underscore}
\usepackage{url}

\usepackage{listings}
\lstset{basicstyle=\footnotesize\ttfamily, breaklines=true, keepspaces=true}

\usepackage{nameref}
\usepackage{amsthm}
\usepackage{enumitem} % continue enumeration 
\usepackage{subfigure}
\usepackage{mdwlist} % compact itemize lists environment

\setcounter{tocdepth}{2}

\renewcommand{\solutiontitle}{\noindent\textbf{Ответ:}\enspace}

\title{Программное моделирование вычислительных систем. Контрольная работа. Вариант \textnumero \variant}
\author{}
\newcommand{\testdate}{} %{\today}
\date{\testdate}

\typeout{Copyright 2015, 2016 Grigory Rechistov.}
\begin{document}

\pagestyle{headandfoot}
\runningheadrule
\firstpageheader{Вариант \textnumero \variant}{Кафедра <<Микропроцессорные технологии>>}{\testdate}
\runningheader{Вариант \textnumero \variant}{Кафедра <<Микропроцессорные технологии>>}{\testdate}
\coverfooter{}{Стр. \thepage\ из \numpages}{}
\firstpagefooter{}{Стр. \thepage\ из \numpages}{}
\runningfooter{}{Стр. \thepage\ из \numpages}{}

\maketitle\thispagestyle{headandfoot} % a hack to have proper footers and headers on the first page as well

\medskip
\makebox[0.9\textwidth]{Ф.И.О.\enspace\hrulefill}
\medskip

\makebox[0.9\textwidth]{Группа\enspace\hrulefill}
\medskip

\makebox[0.9\textwidth]{e-mail\enspace\hrulefill}

% Grade tables
\hqword{Вопрос}
\hpword{Баллов}
\hsword{Результат}
\htword{Сумма}
\vqword{Вопрос}
\vpword{Баллов}
\vsword{Результат}
\vtword{Сумма}
\cellwidth{0.5em}
% \gradetablestretch{1.8}
\begin{center}
\tiny
\gradetable[h][questions]
% \partialgradetable{multiplechoice}[h][questions]\\
% \partialgradetable{fullquestions}[h][questions]\\
% \partialgradetable{bonuswork}[h][questions]\\
\end{center}

\medskip

\begin{questions}


\question[4] Какой из перечисленных ниже типов симуляторов не существует?
\begin{choices}
\choice полноплатформенный программный
\choice полноплатформенный функциональный
\choice полноплатформенный потактовый
\correctchoice потактовый функциональный
\choice потактовый гибридный
\choice потактовый режима приложения
\end{choices}


\question[4] Какие источники информации могут использоваться для создания программных моделей цифровых устройств?
\begin{choices}
\choice Документация для программиста драйверов устройства
\choice Документация микроархитектуры устройства
\choice Исходный код других моделей устройства или аналогичных (предыдущего поколения)
\choice Результаты обратной разработки (reverse engineering) спецификаций из физического образца устройства
\correctchoice Все вышеперечисленные источники
\end{choices}

\question[4] Почему интерпретатор, использующий шитый код, не может быть написан на языке ANSI Cи?
\begin{choices}
\choice На самом деле, он может быть написан и не требует нестандартных расширений
\correctchoice В ANSI Си отсутствует оператор взятия адреса метки
\choice В ANSI Си отсутствует оператор перехода по метке
\choice Компиляторы ANSI Си не могут обрабатывать программы с невложенными переходами
\choice ANSI Си не поддерживает столь большого числа меток внутри процедуры, требуемых для реализации практически важных интерпретаторов
\end{choices}

\question[4] Какой эффект, негативно влияющий на производительность интерпретации, пытается устранить техника шитого кода?
\begin{choices}
\choice Большая частота промахов кэша данных
\choice Большая частота промахов кэша кода
\correctchoice Большая частота промахов предсказателя переходов
\choice Большое число невыровненных доступов к данным
\choice Большая частота возникновения синхронных исключений
\end{choices}

% Some extra blank pages
\newpage
\phantom{Blank page}

\end{questions}
\end{document}

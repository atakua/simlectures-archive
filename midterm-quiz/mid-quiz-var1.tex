% Compile with XeLaTeX only
%
%\documentclass[a4paper, addpoints, answers]{exam}
\documentclass[a4paper, addpoints]{exam}

\pointpoints{б.}{б.}
\bonuspointpoints{б.}{б.}

\def \variant {1}

\usepackage{fontspec}
%\usepackage[utf8]{inputenc} % uncomment for latex2rtf 

\usepackage{xunicode} % some extra unicode support
\usepackage{xltxtra}

\usepackage{amsfonts}
\usepackage{amsmath}

\usepackage{csquotes}

\usepackage{polyglossia}
\defaultfontfeatures{Scale=MatchLowercase, Mapping=tex-text}

\newfontfamily\russianfont{TextBookC} % Гарнитура "Букварная"
\setromanfont{CMU Serif}
\setsansfont{CMU Sans Serif}
\setmonofont{CMU Typewriter Text}

\setdefaultlanguage[spelling=modern]{russian}
\setotherlanguage{english}
\newcommand{\todo}{{\color{red}TODO}\ }
\usepackage{graphicx}

\tolerance=9999 % let the text underfull be ugly as hell, nobody cares.
\emergencystretch=3cm

\usepackage{hyperref}
\hypersetup{colorlinks=true, linkcolor=black, filecolor=black, citecolor=black, urlcolor=black , pdfauthor=Grigory Rechistov, pdftitle=Контрольная работа по курсу <<Программное моделирование вычислительных систем>>}

\usepackage{footnpag}
\usepackage{indentfirst}
\usepackage{underscore}
\usepackage{url}

\usepackage{listings}
\lstset{basicstyle=\footnotesize\ttfamily, breaklines=true, keepspaces=true}

\usepackage{nameref}
\usepackage{amsthm}
\usepackage{enumitem} % continue enumeration 
\usepackage{subfigure}
\usepackage{mdwlist} % compact itemize lists environment

\setcounter{tocdepth}{2}

\renewcommand{\solutiontitle}{\noindent\textbf{Ответ:}\enspace}

\title{Программное моделирование вычислительных систем. Контрольная работа. Вариант \textnumero \variant}
\author{}
\newcommand{\testdate}{\today}
\date{\testdate}

\typeout{Copyright 2015, 2016 Grigory Rechistov.}
\begin{document}

\pagestyle{headandfoot}
\runningheadrule
\firstpageheader{Вариант \textnumero \variant}{Кафедра <<Микропроцессорные технологии>>}{\testdate}
\runningheader{Вариант \textnumero \variant}{Кафедра <<Микропроцессорные технологии>>}{\testdate}
\coverfooter{}{Стр. \thepage\ из \numpages}{}
\firstpagefooter{}{Стр. \thepage\ из \numpages}{}
\runningfooter{}{Стр. \thepage\ из \numpages}{}

\maketitle\thispagestyle{headandfoot} % a hack to have proper footers and headers on the first page as well

\medskip
\makebox[0.9\textwidth]{Ф.И.О.\enspace\hrulefill}
\medskip

\makebox[0.9\textwidth]{Группа\enspace\hrulefill}
\medskip

\makebox[0.9\textwidth]{e-mail\enspace\hrulefill}

% Grade tables
\hqword{Вопрос}
\hpword{Баллов}
\hsword{Результат}
\htword{Сумма}
\vqword{Вопрос}
\vpword{Баллов}
\vsword{Результат}
\vtword{Сумма}
\cellwidth{0.5em}
% \gradetablestretch{1.8}
\begin{center}
\tiny
\gradetable[h][questions]
% \partialgradetable{multiplechoice}[h][questions]\\
% \partialgradetable{fullquestions}[h][questions]\\
% \partialgradetable{bonuswork}[h][questions]\\
\end{center}

\medskip

\begin{questions}

% Introduce several grading range so that grading tables are short enough to fit on the page

% \begingradingrange{multiplechoice} 

% lecture01-why-simulation

% \question[2] Определите понятие <<функциональная симуляция>> в контексте моделирования компьютеров.
% \begin{solution}[2cm]
% \end{solution}
\question[2] Определите понятие <<хозяйская система>> в контексте моделирования компьютеров.
\begin{solution}[2cm]
\end{solution}

% lecture02-common-questions
% \question[2] Объясните, какие сложности традиционной разработки аппаратуры позволяет обойти использование программных симуляторов?
% \begin{solution}[2cm]
% \end{solution}
\question[2] Какие источники информации могут использоваться для создания программной модели цифрового устройства?
\begin{solution}[2cm]
\end{solution}

% lecture03-cpu-interpretation
% \question[2] Какие компоненты модели-интерпретатора ЦПУ удобно генерировать из общего файла описания?
% \begin{solution}[2cm]
% \end{solution}
\question[2] Объясните принцип работы кэширующего (caching) интерпретатора.
\begin{solution}[2cm]
\end{solution}

\question[2] Определите понятие <<синхронное исключение>>. Приведите пример ситуации, когда возникает это событие.
\begin{solution}[2cm]
\end{solution}

%04-arch-state
\question[2] Почему использование стандартных типов языка Си, таких как int, long, long long, не является удачным для представления элементов архитектурного состояния программных моделей? Какие типы следует использовать?
\begin{solution}[2cm]
\end{solution}

\question[2] Опишите, как работает архитектурный механизм Programmable Input/Output (PIO) для доступа к периферийным устройствам.
\begin{solution}[2cm]
\end{solution}

% lecture04-bt-pt1
% lecture04pt2-bt
\question[2] Какие условия ограничивают длину блока трансляции (группы гостевых инструкций) в системах двоичной трансляции?
\begin{solution}[2cm]
\end{solution}

\question[2] Почему набор оптимизаций, применимых при динамической двоичной трансляции, меньше, чем для классической трансляции с языков высокого уровня?
\begin{solution}[2cm]
\end{solution}

% ============================================================
\begingradingrange{bonuswork}
\bonusquestion \textit{Баллы за работу в течение семестра}
\endgradingrange{bonuswork}

% Some extra blank pages
\newpage
\phantom{Blank page}

\end{questions}
\end{document}

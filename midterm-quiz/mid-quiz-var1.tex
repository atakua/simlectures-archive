% Compile with XeLaTeX only
%
%\documentclass[a4paper, addpoints, answers]{exam}
\documentclass[a4paper, addpoints]{exam}

\pointpoints{б.}{б.}
\bonuspointpoints{б.}{б.}

\def \variant {1}

\usepackage{fontspec}
%\usepackage[utf8]{inputenc} % uncomment for latex2rtf 

\usepackage{xunicode} % some extra unicode support
\usepackage{xltxtra}

\usepackage{amsfonts}
\usepackage{amsmath}

\usepackage{csquotes}

\usepackage{polyglossia}
\defaultfontfeatures{Scale=MatchLowercase, Mapping=tex-text}

\newfontfamily\russianfont{TextBookC} % Гарнитура "Букварная"
\setromanfont{CMU Serif}
\setsansfont{CMU Sans Serif}
\setmonofont{CMU Typewriter Text}

\setdefaultlanguage[spelling=modern]{russian}
\setotherlanguage{english}
\newcommand{\todo}{{\color{red}TODO}\ }
\usepackage{graphicx}

\tolerance=9999 % let the text underfull be ugly as hell, nobody cares.
\emergencystretch=3cm

\usepackage{hyperref}
\hypersetup{colorlinks=true, linkcolor=black, filecolor=black, citecolor=black, urlcolor=black , pdfauthor=Grigory Rechistov, pdftitle=Контрольная работа по курсу <<Программное моделирование вычислительных систем>>}

\usepackage{footnpag}
\usepackage{indentfirst}
\usepackage{underscore}
\usepackage{url}

\usepackage{listings}
\lstset{basicstyle=\footnotesize\ttfamily, breaklines=true, keepspaces=true}

\usepackage{nameref}
\usepackage{amsthm}
\usepackage{enumitem} % continue enumeration 
\usepackage{subfigure}
\usepackage{mdwlist} % compact itemize lists environment

\setcounter{tocdepth}{2}

\renewcommand{\solutiontitle}{\noindent\textbf{Ответ:}\enspace}

\title{Программное моделирование вычислительных систем. Контрольная работа. Вариант \textnumero \variant}
\author{}
\newcommand{\testdate}{} %{\today}
\date{\testdate}

\typeout{Copyright 2015, 2016 Grigory Rechistov.}
\begin{document}

\pagestyle{headandfoot}
\runningheadrule
\firstpageheader{Вариант \textnumero \variant}{Кафедра <<Микропроцессорные технологии>>}{\testdate}
\runningheader{Вариант \textnumero \variant}{Кафедра <<Микропроцессорные технологии>>}{\testdate}
\coverfooter{}{Стр. \thepage\ из \numpages}{}
\firstpagefooter{}{Стр. \thepage\ из \numpages}{}
\runningfooter{}{Стр. \thepage\ из \numpages}{}

\maketitle\thispagestyle{headandfoot} % a hack to have proper footers and headers on the first page as well

\medskip
\makebox[0.9\textwidth]{Ф.И.О.\enspace\hrulefill}
\medskip

\makebox[0.9\textwidth]{Группа\enspace\hrulefill}
\medskip

\makebox[0.9\textwidth]{e-mail\enspace\hrulefill}

% Grade tables
\hqword{Вопрос}
\hpword{Баллов}
\hsword{Результат}
\htword{Сумма}
\vqword{Вопрос}
\vpword{Баллов}
\vsword{Результат}
\vtword{Сумма}
\cellwidth{0.5em}
% \gradetablestretch{1.8}
\begin{center}
\tiny
\gradetable[h][questions]
% \partialgradetable{multiplechoice}[h][questions]\\
% \partialgradetable{fullquestions}[h][questions]\\
% \partialgradetable{bonuswork}[h][questions]\\
\end{center}

\medskip

\begin{questions}


\question[4] Какая аналогия наилучшим образом описывает понятие «функциональная симуляция»?
\begin{choices}
\correctchoice чёрный ящик
\choice прозрачный ящик
\choice серая коробка
\choice стеклянный ящик
\choice матрёшка из ящиков
\end{choices}

\question[4] Расставьте типы симуляторов в порядке убывания типичной скорости симуляции, от самого быстрого до самого медленного.
\begin{choices}
\choice Потактовая
\choice Функциональная с интерпретацией
\choice Функциональная с двоичной трансляцией
\end{choices}
\begin{solution}
    ДТ, интерпретатор, потактовый.
\end{solution}


\question[4] Какие типы операндов может иметь машинная инструкция?
\begin{choices}
\choice явные (закодированные в машинном коде конкретной команды) входные (source)
\choice явные выходные (destination)
\choice неявные входные
\choice неявные выходные
\choice явные входные-выходные
\choice неявные входные-выходные
\correctchoice все вышеперечисленные варианты
\end{choices}

\question[4] Что такое \textit{ленивое вычисление} выражения?
\begin{choices}
\choice Стратегия выполнения, при которой вычисление выражения никогда не совершается
\choice Стратегия выполнения, при которой вычисление выражения никогда не заканчивается
\choice Стратегия выполнения, при которой результат вычисления выражения кэшируется для последующих использований
\correctchoice Стратегия выполнения, при которой вычисление выражения происходит, только если его результат используется где-то ещё в программе
\choice Все вышеперечисленные стратегии являются вариантами ленивого вычисления
\end{choices}



% Some extra blank pages
\newpage
\phantom{Blank page}

\end{questions}
\end{document}

% Compile with XeLaTeX, TeXLive 2013 or more recent
\documentclass{beamer}

% Base packages
\usepackage{fontspec}
\usepackage{xunicode}
\usepackage{xltxtra}

\usepackage{amsfonts}
\usepackage{amsmath}
\usepackage{longtable}
\usepackage{csquotes}
\usepackage{standalone}
\usepackage{bytefield}

% Setup fonts
\newfontfamily\russianfont{CMU Serif}
\setromanfont{CMU Serif}
\setsansfont{CMU Sans Serif}
\setmonofont{CMU Typewriter Text}

% Setup Russian hyphenation. NOTE: this declaration *must* come after fontspec's font declarations,
% or a mysterious (but harmless in other respects) error "Improper `at' size (0.0pt), replaced by 10pt." would appear.
\usepackage{polyglossia}
\defaultfontfeatures{Scale=MatchLowercase, Mapping=tex-text}

\setdefaultlanguage[spelling=modern]{russian} % for polyglossia
\setotherlanguage{english} % for polyglossia

% Vector drawings
\usepackage{tikz}
% \usetikzlibrary{shapes, calc, arrows, fit, positioning, decorations, patterns, decorations.pathreplacing, chains, snakes}
\usetikzlibrary{shapes, calc, arrows, decorations.markings, decorations.pathmorphing, decorations, patterns, chains, snakes, backgrounds, positioning, fit, petri}
\usepackage[siunitx]{circuitikz}

% Be able to insert hyperlinks
\usepackage{hyperref}
\hypersetup{colorlinks=true, linkcolor=black, filecolor=black, citecolor=black, urlcolor=blue , pdfauthor=Grigory Rechistov <grigory.rechistov@phystech.edu>, pdftitle=Курс «Программное моделирование вычислительных систем»}
% \usepackage{url}

% Misc optional packages
\usepackage{underscore}
\usepackage{amsthm}

% A new command to mark not done places
\newcommand{\todo}[1][Напиши меня]{{\color{red}TODO\ #1}}
\newcommand{\abbr}{\textit{англ.}\ }

\title{Двоичная трансляция}
\subtitle{Курс «Программное моделирование вычислительных систем»}
\subject{Курс «Программное моделирование вычислительных систем»}

\author[]{Григорий Речистов \\ \small{\href{mailto:grigory.rechistov@phystech.edu}{grigory.rechistov@phystech.edu}}}
\date{\today}
\pgfdeclareimage[height=0.5cm]{ilab-logo}{../ilab-noletters.png}
\logo{\pgfuseimage{ilab-logo}}

\typeout{Copyright 2015 Grigory Rechistov}

\usetheme{Berlin}
\setbeamertemplate{navigation symbols}{}%remove navigation symbols

\begin{document}

\begin{frame}
    \maketitle
\end{frame}

\begin{frame}
    \tableofcontents
\end{frame}

\section*{Обзор}

\begin{frame}{На (поза)прошлой лекции}
\begin{itemize}
\item Интерпретаторы — медленная шутка
\item Рассмотренные улучшения основаны на повторном использовании уже полученных результатов
\item Существуют устоявшиеся идиомы для представления моделируемого архитектурного состояния
\end{itemize}
\end{frame}

\begin{frame}{Вопросы}
\begin{itemize}
\item Определите термин «декодирование» \pause
\item Сколько бит в машинном слове? \pause
\item Что лучше — MMIO или PIO?
\end{itemize}

\end{frame}

% «»

\begin{frame}{Что удалось соптимизировать в интерпретаторе}
\begin{itemize}
\item Fetch \todo{appearify}
\item Decode
\item Execute
\item Writeback
\item Advance PC
\end{itemize}
\end{frame}

\section{Компиляция и трансляция}

\begin{frame}{Интерпретация и трансляция в языках высокого уровня}
\begin{itemize}
\item Basic, CPython, Shell
    \begin{itemize}
    \item Прочитать строку - распознать команды - исполнить
    \item Медленно, но больше «интерактивности»
    \end{itemize}
\item Fortran, C, Pascal
    \begin{itemize}
    \item Первый проход: распознавание команд и преобразование их в машинный код
    \item Второй проход: исполнение машинного кода
    \end{itemize}
\end{itemize}
\end{frame}

\begin{frame}{Двоичная трансляция}
\begin{itemize}
\item Входной язык - гостевой машинный код
\item Целевой язык - хозяйский машинный код
\item ДТ - перевод года гостевой программы, записанной в гостевой ISA, в эквивалентный код в терминах хозяйской ISA.
\item Ради чего: многократное исполнение результатов трансляции
\item \todo вопрос: что такое декомпиляция? (маш. код в ЯВО)
\end{itemize}
\end{frame}

\begin{frame}[fragile]{Фазы ДТ}
\centering
\input{./../../simbook/metoda/drawings/dynamic-bt}
\end{frame}

\section{Алгоритмы}

\begin{frame}{Алгоритм 1 - шаблонная трансляция}
\todo
% \begin{itemize}
translate() {
PC = start_addr;
bufptr = start_buf;
while (! enough) {
instr = fetch(PC);
opcode = decode(instr);
capsule = capsules[opcode];
memcpy(capsule, bufptr);
PC ++;
bufptr ++;
};
memcpy(return_jmp, bufptr);
};
% \end{itemize}
\end{frame}


\begin{frame}{Капсула}
\centering
\input{./../../simbook/metoda/drawings/capsule}

\end{frame}

\begin{frame}[fragile]{Ленивое выделение памяти}
Для непрерывного дипазона адресов хранилище выделяется по мере необходимости кусками фиксированного размера (страницами).

\begin{verbatim}
page = addr & PAGE_MASK;
hptr = lookup_hptr(page);
if (!hptr)
    hptr = allocate_hptr(page);
assert(hptr);
haddr = hptr + (addr & PAGE_OFFSET);
\end{verbatim}
При недостатке хозяйской памяти «старые» страницы выгружаются на диск.
\end{frame}

\begin{frame}[fragile]{Звучит знакомо?}
\begin{itemize}
\item Это же виртуальная память!
\item POSIX-системы предоставляют механизм mmap()
\end{itemize}

\begin{verbatim}
void *mmap(void *addr, 
    size_t len, int prot, int flags,
    int fildes, off_t off); 
\end{verbatim}

\centering

\begin{tikzpicture}[>=latex, font=\small, node distance=0mm, inner xsep=0pt, text width = 10mm, every node/.style=draw]

\begin{scope}[start chain]
    \node [on chain] (vp0) {};
    \node [on chain] {};
    \node [fill=black!10, on chain] (vp2) {};
    \node [fill=black!10, on chain] (vp3) {};
    \node [on chain] {};
    \node [on chain] {};
    \node [on chain] {};    
    \node [fill=black!10, on chain] (vp4) {};    
    \node [fill=black!10, on chain] (vp1) {};    
    \node [on chain] {};    
    \node [fill=black!10, on chain] (vp5) {};    
\end{scope}

\begin{scope}[start chain]
    \node [on chain, below=1cm of vp0] (pp1) {} ;
    \node [on chain] (pp2) {};
    \node [on chain] (pp3) {};
    \node [on chain] (pp4) {};

\end{scope}
    \node [right=2cm of pp4] (dp1) {};    

\draw[->] (vp1.south) -- (pp1.north);
\draw[->] (vp2.south) -- (pp2.north);
\draw[->] (vp3.south) -- (pp3.north);
\draw[->] (vp4.south) -- (pp4.north);
\draw[->] (vp5.south) -- (dp1.north);
    
\node[draw=none, above=0.1cm of vp0.east, text width=4cm, xshift=1cm] {Выделенный блок};
\node[draw=none, below=0.1cm of pp1.east, text width=4cm, xshift=1cm] {Физические страницы};
\node[draw=none, below=0.1cm of dp1.east, text width=4cm, xshift=1cm] {Дисковый своп};


\end{tikzpicture}

    % \path (0,0.5) coordinate (a1);
    % \draw (0,0) rectangle (7.5,1);
    % \node[left = 0.25cm of a1] (host-disk) {Массив};
    
    % \draw (3,-2) rectangle (7,-1);
    % \node[below = 1.5cm of host-disk] {Гостевой диск};
        
    % \path[thick, fill=black!25] (1.2,0) rectangle (1.49,1);
    % \path[fill=black!20] (1.51,0) rectangle (2.49,1);
    % \path[fill=black!20] (2.51,0) rectangle (3.5,1);
    
    % \path[fill=black!20] (4.,-2) rectangle (4.99,-1);
    % \path[fill=black!20] (5.01,-2) rectangle (6,-1);
    
    % \draw[<->] (4.25,-1) -- (2,0);
    % \draw[<->] (5.25,-1) -- (3,0);
    
    % \path (4.5,-2) coordinate (a2);
    % \node[below =0.25cm of a2, inner sep=1pt] (used) {Используемые области};
    
    % \draw[dotted] (used) -- (4.25,-2);
    % \draw[dotted] (used) -- (5.25,-2);
    
    % \draw[decorate, decoration={brace, amplitude=3pt}, yshift=3pt] (1.2,1) -- (3.5,1) coordinate[midway] (a3);
    % \path (1.35,1) coordinate (a4);
    % \node[above = 0.1cm of a3] {Образ диска};    
    


\end{frame}

\section{Регистры, поля, банки}

\begin{frame}[fragile]{Два способа взаимодействия с регистрами устройств}
PIO — programmable I/O, выделенные инструкции для общения с периферией

\begin{verbatim}
IN EAX, DX
OUT DX, EAX
\end{verbatim}
\pause

MMIO — memory mapped I/O, унифицированный подход к доступу к ОЗУ и устройствам

\begin{verbatim}
MOV [mem], reg
MOV reg, [mem]
\end{verbatim}

\end{frame}

\begin{frame}{Операции над регистрами}
\begin{itemize}
\item read, write, fetch, prefetch
\item Регистры-хранилище и регистры с побочными эффектами
\item Примеры регистров с side-effects: time stamp (RW), command (W), status (R), version (RO)
\item inquiry — «призрачное» обращение (без эффектов)
\item reset
\end{itemize}

\end{frame}

\begin{frame}[fragile]{Операции над регистрами}
\begin{verbatim}
template <class rtype> class IRegister {
  virtual Exception Read(rtype& retval) = 0;
  virtual Exception Fetch(rtype& retval) = 0;
  virtual Exception Prefetch(rtype& retval) = 0;
  virtual Exception Write(const rtype& newval) = 0;
  virtual bool InquiryRead(rtype& retval) = 0;
  virtual bool InquiryWrite(const rtype& newval) = 0;
  virtual void Reset() = 0;
}
\end{verbatim}
\end{frame}

\begin{frame}[fragile]{Битовые поля}

\begin{bytefield}[bitwidth=0.8em, endianness=big]{32}
\bitheader{31,19,18,17,16,15,13,12,11,8,7,0} \\
\bitbox{13}{}&\bitbox{2}{\tiny{mode}} & \bitbox{1}{\rotatebox{90}{\tiny{mask}}} & \bitbox{3}{} & \bitbox{1}{\rotatebox{90}{\tiny{status}}} & \bitbox{4}{} & \bitbox{8}{vector} \\
\end{bytefield}

\tiny{APIC LVT Timer. Intel® 64 and IA-32 Architectures Software Developer’s Manual}

\normalsize
\begin{itemize}
\item Настоящие базовые единицы спецификаций и моделирования
\item Могут иметь совершенно различные свойства внутри регистра
\end{itemize}

\end{frame}

\begin{frame}[fragile]{Банк регистров}
Группа регистров устройства, находящиеся в одной области
\vfill
\centering

\begin{tikzpicture}[font=\small, >=latex]

\node[text height=5cm, text width=1cm, draw]  (physspace) {

};

\node[right=of physspace] (bank) {
\begin{bytefield}[bitwidth=.5em, endianness=big]{32}
\bitheader{31, 0}\\
\wordbox{1}{reg_base}\\
\wordbox{1}{reg_offset}\\
\bitbox{25}{reg_cmd_in} & \bitbox{7}{rsvd}\\
\wordbox{2}{reg_data_out}\\
\end{bytefield}
};

\draw (physspace.10) -- ([yshift=-0.25cm]bank.north west);
\draw (physspace.350) -- ([yshift=0.5cm]bank.south west);

\node[left=0cm of physspace.north west] {\texttt{0x0}};
\node[left=0cm of physspace.south west] {\texttt{0xffff}};

\end{tikzpicture}

\end{frame}
\begin{frame}{Карты памяти}
\input{./../../simbook/metoda/drawings/memmap}
% \begin{itemize}
% \item 
% \item 
% \end{itemize}
\end{frame}

\begin{frame}{Пример: devmgmt.msc}
\centering

\end{frame}

\section{Прерывания}

\begin{frame}{Прерывания}
\begin{centering}
\input{./../../simbook/metoda/drawings/interrupt-line}
\end{centering}

\begin{itemize}
\item Как моделировать отдельное прерывание — очень просто:\\
\texttt{take_interrupt(dev_t *dev);}
\item Когда моделировать — вопрос сложнее, тема отдельной лекции
\end{itemize}

\end{frame}

\begin{frame}{Преобразование адресов}
\begin{itemize}
\item v2p
\item p2h
\item v2p + p2h =  v2h
% \item softTLB
\end{itemize}
\end{frame}

\section{Endianness}

\begin{frame}{Порядок байт при доступах}



\end{frame}

\begin{frame}{Бит, байт, слово}
\begin{itemize}
\item Бит — ?\pause
\item Байт \pause — минимальная адресуемая (в данной архитектуре) единица хранения
информации \pause
\item Октет \pause — восемь бит \pause
\item Машинное слово \pause — максимальный объём информации, который ЦПУ может обработать единовременно
\end{itemize}

\pause
Intel: word — 16 бит, dword — 32 бит, qword — 64 бит.
\end{frame}



\section*{Литература}

\begin{frame}[allowframebreaks]{Литература}
\begin{thebibliography}{99}
    \bibitem{bochsunderhood} Stanislav Shwartsman, Darek Mihoka. How Bochs Works Under the Hood. 2nd edition.
    \url{http://bochs.sourceforge.net/How the Bochs works under the hood 2nd edition.pdf}
    
    \bibitem{bic} M. Domeika, M. Loenko, P. Ozhdikhin, E. Brevnov.  Bi-Endian Compiler: A Robust and High Performance Approach for Migrating Byte Order Sensitive Applications 
    \url{http://world-comp.org/p2011/ESA2902.pdf}
    
\end{thebibliography}
\end{frame}

\section*{Конец}
% The final "thank you" frame 

\begin{frame}{На следующей лекции}

Ещё быстрее! Прямое исполнение

\centering
\includegraphics[width=0.5\textwidth]{./moar}
\end{frame}

\begin{frame}

{\huge{Спасибо за внимание!}\par}

\vfill

Слайды и материалы курса доступны по адресу \url{http://is.gd/ivuboc} % http://atakua.doesntexist.org/wordpress/simulation-course-russian/

\vfill

\tiny{\textit{Замечание}: все торговые марки и логотипы, использованные в данном материале, являются собственностью их владельцев. Представленная здесь точка зрения отражает личное мнение автора, не выступающего от лица какой-либо организации.}

\end{frame}

\end{document}




% Compile with XeLaTeX, TeXLive 2013 or more recent
\documentclass{beamer}

% Base packages
\usepackage{fontspec}
\usepackage{xunicode}
\usepackage{xltxtra}

\usepackage{amsfonts}
\usepackage{amsmath}
\usepackage{longtable}
\usepackage{csquotes}
\usepackage{standalone}
\usepackage{bytefield}

% Setup fonts
\newfontfamily\russianfont{CMU Serif}
\setromanfont{CMU Serif}
\setsansfont{CMU Sans Serif}
\setmonofont{CMU Typewriter Text}

% Setup Russian hyphenation. NOTE: this declaration *must* come after fontspec's font declarations,
% or a mysterious (but harmless in other respects) error "Improper `at' size (0.0pt), replaced by 10pt." would appear.
\usepackage{polyglossia}
\defaultfontfeatures{Scale=MatchLowercase, Mapping=tex-text}

\setdefaultlanguage[spelling=modern]{russian} % for polyglossia
\setotherlanguage{english} % for polyglossia

% Vector drawings
\usepackage{tikz}
% \usetikzlibrary{shapes, calc, arrows, fit, positioning, decorations, patterns, decorations.pathreplacing, chains, snakes}
\usetikzlibrary{shapes, calc, arrows, decorations.markings, decorations.pathmorphing, decorations, patterns, chains, snakes, backgrounds, positioning, fit, petri}
\usepackage[siunitx]{circuitikz}

% Be able to insert hyperlinks
\usepackage{hyperref}
\hypersetup{colorlinks=true, linkcolor=black, filecolor=black, citecolor=black, urlcolor=blue , pdfauthor=Grigory Rechistov <grigory.rechistov@phystech.edu>, pdftitle=Курс «Программное моделирование вычислительных систем»}
% \usepackage{url}

% Misc optional packages
\usepackage{underscore}
\usepackage{amsthm}

% A new command to mark not done places
\newcommand{\todo}[1][Напиши меня]{{\color{red}TODO\ #1}}
\newcommand{\abbr}{\textit{англ.}\ }

\title{Двоичная трансляция}
\subtitle{Курс «Программное моделирование вычислительных систем»}
\subject{Курс «Программное моделирование вычислительных систем»}

\author[]{Григорий Речистов \\ \small{\href{mailto:grigory.rechistov@phystech.edu}{grigory.rechistov@phystech.edu}}}
\date{\today}
\pgfdeclareimage[height=0.5cm]{ilab-logo}{../ilab-noletters.png}
\logo{\pgfuseimage{ilab-logo}}

\typeout{Copyright 2015 Grigory Rechistov}

\usetheme{Berlin}
\setbeamertemplate{navigation symbols}{}%remove navigation symbols

\begin{document}

\begin{frame}
    \maketitle
\end{frame}

\begin{frame}
    \tableofcontents
\end{frame}

\section*{Обзор}

\begin{frame}{На (поза)прошлой лекции}
\begin{itemize}
\item Интерпретаторы — медленная шутка
\item Рассмотренные улучшения основаны на повторном использовании уже полученных результатов
\item Существуют устоявшиеся идиомы для представления моделируемого архитектурного состояния
\end{itemize}
\end{frame}

\begin{frame}{Вопросы}
\begin{itemize}
\item Определите термин «декодирование» \pause
\item Сколько бит в машинном слове? \pause
\item Что лучше — MMIO или PIO?
\end{itemize}

\end{frame}

% «»

\begin{frame}{Что удалось соптимизировать в интерпретаторе}
\begin{itemize}
\item Fetch \todo{appearify}
\item Decode
\item Execute
\item Writeback
\item Advance PC
\end{itemize}
\end{frame}

\section{Компиляция и трансляция}

\begin{frame}{Интерпретация и трансляция в языках высокого уровня}
\begin{itemize}
\item Basic, CPython, Shell
    \begin{itemize}
    \item Прочитать строку — распознать команды — исполнить
    \item Медленно, но больше «интерактивности»
    \end{itemize}
\item Fortran, C, Pascal
    \begin{itemize}
    \item Первый проход: распознавание команд и преобразование их в машинный код
    \item Второй проход: исполнение машинного кода
    \end{itemize}
\end{itemize}
\end{frame}

\begin{frame}{Двоичная трансляция}
\begin{itemize}
\item Входной язык — гостевой машинный код
\item Целевой язык — хозяйский машинный код
\item ДТ - перевод года гостевой программы, записанной в гостевой ISA, в эквивалентный код в терминах хозяйской ISA.
\item Ради чего: многократное исполнение результатов трансляции
\item \todo вопрос: что такое декомпиляция? (маш. код в ЯВО)
\end{itemize}
\end{frame}

\begin{frame}[fragile]{Фазы ДТ}
\centering
\input{./../../simbook/metoda/drawings/dynamic-bt}
\end{frame}

\section{Алгоритмы}

\begin{frame}{Алгоритм 1: шаблонная трансляция}
\todo
% \begin{itemize}
translate() {
PC = start_addr;
bufptr = start_buf;
while (! enough) {
instr = fetch(PC);
opcode = decode(instr);
capsule = capsules[opcode];
memcpy(capsule, bufptr);
PC ++;
bufptr ++;
};
memcpy(return_jmp, bufptr);
};
% \end{itemize}
\end{frame}


\begin{frame}{Капсула}
\begin{itemize}
    \item Гостевой код, архитектура IA-32 EMT (64 бит)
    \item Хозяйский код, архитектура IA-32 (32 бит)
\end{itemize}

\centering
\input{./../../simbook/metoda/drawings/capsule}

\end{frame}


\begin{frame}{Блоки трансляции}
\centering
\input{./../../simbook/metoda/drawings/bb-translation}
% \begin{itemize}
% \item
% \item
% \item
% \item 
% \end{itemize}


\end{frame}

\begin{frame}{Алгоритм 2: just-in-time}
    \todo
\end{frame}

\begin{frame}{Статическая и динамическая ДТ}
\todo
\end{frame}

\begin{frame}{Оптимизации}
\centering
\input{./../../simbook/metoda/drawings/bt-optimization}

\end{frame}

\begin{frame}{Почему оптимизации при ДТ затруднительны}

В отличие от ЯВО, машинный код содержит
меньше информации об исходном алгоритме
Мы не можем делать многие предположения,
необходимые для компиляторных оптимизаций
без нарушения корректности
–
 Адреса переменных — их нет
–
 Границы процедур — их нет
–
 Адреса переходов — известна только часть из них
\end{frame}

\begin{frame}{Самомодифицирующийся код(self modifying code, SMC)}

\end{frame}

\begin{frame}{Обнаружение кода 1}

\end{frame}

\begin{frame}{Обнаружение кода 2}
\input{./../../simbook/metoda/drawings/byte-shift}

\end{frame}

\begin{frame}{Опережающая трансляция}

\end{frame}

\begin{frame}{Трансляция горячего кода}

\end{frame}

\begin{frame}{Время компиляции и время исполнения}
\todo новое
\end{frame}

\begin{frame}{Переключение таблиц}
\todo новое
\end{frame}


\begin{frame}{Итоги}
\begin{itemize}
\item Интерпретация, компиляция (трансляция)
\item Двоичная трансляция. Статическая, динамическая трансляция
\item Капсула
\item SMC
\item Code discovery
\item (Не)возможность оптимизации кода при ДТ
\end{itemize}
\end{frame}


\section*{Литература}

\begin{frame}[allowframebreaks]{Литература}
\begin{thebibliography}{99}
    \bibitem{vms} Jim Smith, Ravi Nair. Virtual Machines: Versatile Platforms for Systems and Processes. 2005
    \bibitem{qemu} Fabrice Bellard. QEMU, a Fast and Portable Dynamic Translator \url{http://www.usenix.org/publications/library/proceedings/usenix05/tech/freenix/full_papers/bellard/bellard.pdf}

    
\end{thebibliography}
\end{frame}

\section*{Конец}
% The final "thank you" frame 

\begin{frame}{На следующей лекции}

Ещё быстрее! Прямое исполнение

\centering
\includegraphics[width=0.5\textwidth]{./moar}
\end{frame}

\begin{frame}

{\huge{Спасибо за внимание!}\par}

\vfill

Слайды и материалы курса доступны по адресу \url{http://is.gd/ivuboc} % http://atakua.doesntexist.org/wordpress/simulation-course-russian/

\vfill

\tiny{\textit{Замечание}: все торговые марки и логотипы, использованные в данном материале, являются собственностью их владельцев. Представленная здесь точка зрения отражает личное мнение автора, не выступающего от лица какой-либо организации.}

\end{frame}

\end{document}




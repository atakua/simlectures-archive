% Compile with XeLaTeX, TeXLive 2013 or more recent
\documentclass{beamer}

% Base packages
\usepackage{fontspec}
\usepackage{xunicode}
\usepackage{xltxtra}

\usepackage{amsfonts}
\usepackage{amsmath}
\usepackage{longtable}
\usepackage{csquotes}
\usepackage{standalone}

% Setup fonts
\newfontfamily\russianfont{CMU Serif}
\setromanfont{CMU Serif}
\setsansfont{CMU Sans Serif}
\setmonofont{CMU Typewriter Text}

% Setup Russian hyphenation. NOTE: this declaration *must* come after fontspec's font declarations,
% or a mysterious (but harmless in other respects) error "Improper `at' size (0.0pt), replaced by 10pt." would appear.
\usepackage{polyglossia}
\defaultfontfeatures{Scale=MatchLowercase, Mapping=tex-text}

\setdefaultlanguage[spelling=modern]{russian} % for polyglossia
\setotherlanguage{english} % for polyglossia

% Vector drawings
\usepackage{tikz}
\usetikzlibrary{shapes, calc, arrows, fit, positioning, decorations, patterns, decorations.pathreplacing, chains, snakes}
\usepackage[siunitx]{circuitikz}

\graphicspath{./../../simbook/metoda/pictures/}

% Be able to insert hyperlinks
\usepackage{hyperref}
\hypersetup{colorlinks=true, linkcolor=black, filecolor=black, citecolor=black, urlcolor=blue , pdfauthor=Grigory Rechistov <grigory.rechistov@phystech.edu>, pdftitle=Курс «Программное моделирование вычислительных систем»}
% \usepackage{url}

% Misc optional packages
\usepackage{underscore}
\usepackage{amsthm}

% A new command to mark not done places
\newcommand{\todo}[1][Напиши меня]{{\color{red}TODO\ #1}}

\title{Роль моделирования в процессе разработки программно-аппаратных платформ}
\subtitle{Курс «Программное моделирование вычислительных систем»}
\subject{Курс «Программное моделирование вычислительных систем»}

\author[Григорий Речистов]{Григорий Речистов \\ \small{\href{mailto:grigory.rechistov@phystech.edu}{grigory.rechistov@phystech.edu}}}
\date{\today}
\pgfdeclareimage[height=0.5cm]{ilab-logo}{../ilab-noletters.png}
\logo{\pgfuseimage{ilab-logo}}

\typeout{Copyright 2015 Grigory Rechistov}

\usetheme{Berlin}
\setbeamertemplate{navigation symbols}{}%remove navigation symbols

\begin{document}

\begin{frame}
    \maketitle
\end{frame}

\begin{frame}
    \tableofcontents
\end{frame}

\section{Сложность вычислительных систем}

\begin{frame}{Сложность вычислительных систем}

\end{frame}

\begin{frame}{Почему разработка только на реальном железе невыгодна }

\begin{itemize}
\item Количество доступных образцов мало
\item Отладка очень сложна
\item Цикл разработки длинный
\end{itemize}

Стоимость разработки увеличивается.

I've noticed a shift during the past couple of years towards an increasing use of various types of simulation, including virtual platforms. Previously software developers wanted real hardware, but now they have to start using simulation because there's no chip available. Tomas Evensen, Wind River CTO.

\end{frame}

\begin{frame}
\centering

\includegraphics[width=0.8\textwidth]{ic-floor}


\end{frame}

\begin{frame}{Программные модели}
\centering 
\include{./../../simbook/metoda/drawings/idea}

\end{frame}


\begin{frame}{Области использования}

\begin{itemize}
\item Новое аппаратное обеспечение
\item Совместная разработка аппаратуры и ПО
\item Экспериментальные архитектуры
\item Предсказание производительности, потребления мощности
\item Обеспечение совместимости с другими архитектурами
\end{itemize}

\end{frame}


\begin{frame}{Симуляторы на различных этапах разработки}

\include{./../../simbook/metoda/drawings/error-cost}

\end{frame}

\begin{frame}{Области использования}

\end{frame}

\begin{frame}{Области использования}

\end{frame}

\begin{frame}{Области использования}

\end{frame}

\begin{frame}{Области использования}

\end{frame}

\section{Литература}

\begin{frame}[allowframebreaks]{Литература}
\begin{thebibliography}{99}
    \bibitem{simbook} Программное моделирование вычислительных систем (лекции). \url{http://atakua.doesntexist.org/public/archive/simcourse/simulation-lectures-latest.pdf}

    \bibitem{practicum} Лабораторный практикум по программному моделированию.  \url{http://atakua.doesntexist.org/public/archive/simcourse/simulation-practicum-latest.pdf}

\end{thebibliography}
\end{frame}


\section{Конец}
% The final "thank you" frame
\begin{frame}

{\huge{Спасибо за внимание!}\par}

\vfill

Слайды и материалы курса доступны по адресу \url{http://is.gd/ivuboc} % http://atakua.doesntexist.org/wordpress/simulation-course-russian/

\vfill

\tiny{\textit{Замечание}: все торговые марки и логотипы, использованные в данном материале, являются собственностью их владельцев. Представленная точка зрения отражает личное мнение автора.
%Материалы доступны по лицензии Creative Commons Attribution-ShareAlike (Атрибуция — С сохранением условий) 4.0 весь мир (в т.ч. Россия и др.). Чтобы ознакомиться с экземпляром этой лицензии, посетите \url{http://creativecommons.org/licenses/by-sa/4.0/}
}

\end{frame}


\end{document}

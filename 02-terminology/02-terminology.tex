% Compile with XeLaTeX, TeXLive 2013 or more recent
\documentclass{beamer}

% Base packages
\usepackage{fontspec}
\usepackage{xunicode}
\usepackage{xltxtra}

\usepackage{amsfonts}
\usepackage{amsmath}
\usepackage{longtable}
\usepackage{csquotes}
\usepackage{standalone}

% Setup fonts
\newfontfamily\russianfont{CMU Serif}
\setromanfont{CMU Serif}
\setsansfont{CMU Sans Serif}
\setmonofont{CMU Typewriter Text}

% Setup Russian hyphenation. NOTE: this declaration *must* come after fontspec's font declarations,
% or a mysterious (but harmless in other respects) error "Improper `at' size (0.0pt), replaced by 10pt." would appear.
\usepackage{polyglossia}
\defaultfontfeatures{Scale=MatchLowercase, Mapping=tex-text}

\setdefaultlanguage[spelling=modern]{russian} % for polyglossia
\setotherlanguage{english} % for polyglossia

% Vector drawings
\usepackage{tikz}
\usetikzlibrary{shapes, calc, arrows, fit, positioning, decorations, patterns, decorations.pathreplacing, chains, snakes}
\usepackage[siunitx]{circuitikz}

\graphicspath{./../../simbook/metoda/pictures/}

% Be able to insert hyperlinks
\usepackage{hyperref}
\hypersetup{colorlinks=true, linkcolor=black, filecolor=black, citecolor=black, urlcolor=blue , pdfauthor=Grigory Rechistov <grigory.rechistov@phystech.edu>, pdftitle=Курс «Программное моделирование вычислительных систем»}
% \usepackage{url}

% Misc optional packages
\usepackage{underscore}
\usepackage{amsthm}

% A new command to mark not done places
\newcommand{\todo}[1][Напиши меня]{{\color{red}TODO\ #1}}

\title{Общие свойства симуляторов}
\subtitle{Курс «Программное моделирование вычислительных систем»}
\subject{Курс «Программное моделирование вычислительных систем»}

\author[]{Григорий Речистов \\ \small{\href{mailto:grigory.rechistov@phystech.edu}{grigory.rechistov@phystech.edu}}}
\date{\today}
\pgfdeclareimage[height=0.5cm]{ilab-logo}{../ilab-noletters.png}
\logo{\pgfuseimage{ilab-logo}}

\typeout{Copyright 2015 Grigory Rechistov}

\usetheme{Berlin}
\setbeamertemplate{navigation symbols}{}%remove navigation symbols

\begin{document}

\begin{frame}
    \maketitle
\end{frame}

\begin{frame}
    \tableofcontents
\end{frame}

\section{Требования}

\begin{frame}{На прошлой лекции}
Симуляторы
\begin{itemize}
\item сократили длительность цикла проектирования вычислительных систем
\item увеличили эффективность труда проектировщика
\item стали незаменимы при совместной разработке аппаратуры и программного обеспечения
\end{itemize}
\end{frame}

\begin{frame}{Вопросы}
\begin{itemize}
\item Что такое функциональный симулятор режима приложения?\pause 
\item Почему программы и аппаратуру необходимо разрабатывать совместно?\pause
\item Может ли Dosbox загрузить Windows? \pause Да, если это Win 3.1
\end{itemize}
\end{frame}


\begin{frame}{Требования к симуляторным решениям}
\begin{itemize}
\item Точность
\item Скорость
\item Совместимость
\end{itemize}
\end{frame}

\section{Точность}

\begin{frame}{Точность}

\begin{flushright}
\tiny{Essentially, all models are wrong, but some are useful. \textit{G. E. P. Box, N. R. Draper}}
\end{flushright}

\begin{itemize}
\item Точность должна быть достаточной для выполнения целей, поставленных перед симулятором
\item Излишняя точность — медленная работа, дольше разработка, больше ошибок
\end{itemize}

\end{frame}

\begin{frame}{Откуда берутся данные для создания модели}

\begin{itemize}
\item По спецификациям, документации: SDM, EAS, BWG\dots
\item Из предыдущей модели + спецификации
\item Из анализа работы реальной аппаратуры (reverse engineering)
\end{itemize}

\end{frame}

\begin{frame}{Как проверить корректность модели?}

\begin{itemize}
\item Функциональная корректность
\item Адекватность результатов
\item Сравнение с другой моделью этого же устройства
\item Сравнение с реальной аппаратурой
\end{itemize}
\end{frame}

\begin{frame}{Как \textit{доказать} корректность модели?}

\begin{itemize}
\item Формальная верификация
\item Не всегда возможна
\item Огромное число состояний: длина одной инструкции IA-32 до 15 байт, а это $2^{120}$ комбинаций входных значений
\end{itemize}
\end{frame}

\section{Скорость}

\begin{frame}{Скорость}
\begin{itemize}
\item MIPS — миллионы гостевых инструкций, исполненных за одну хозяйскую секунду
\item МГц — число гостевых тактов, исполненных за одну хозяйскую секунду
\end{itemize}
\end{frame}

\begin{frame}{Скорость как замедление}
Slowdown. Сравнение с физической системой со сходными параметрами и исполняющей аналогичный сценарий
\begin{itemize}
\item Необходима реальная система, с которой можно сравниваться
\item Специальный случай: целевая архитектура = хозяйская % «»
\item Чаще всего модель работает медленнее аппаратуры
\end{itemize}
\end{frame}

\begin{frame}{Замедление разных типов симуляции}
\begin{tabular}{rl}
\textbf{Тип модели}                    & \textbf{Замедление} \\
Функциональная с аппаратным ускорением & $1$\dots$5$ \\
Функциональная с двоичной трансляцией  & $10$\dots$100$ \\
Функциональная с интерпретацией        & $100$\dots$10^3$ \\
Потактовая                             & $10^3$\dots$10^6$ \\
\end{tabular}

\end{frame}


\begin{frame}{Как повысить скорость симулятора?}
\begin{itemize}
\item Это мы будем обсуждать на многих последующих занятиях
\item Эффективная симуляция и бездействие, аппаратная поддержка, параллельное исполнение\dots
\end{itemize}
\end{frame}

\section{Совместимость}

\begin{frame}{Совместимость}
\begin{itemize}
\item Отладчики, среды разработки, анализаторы трасс
\item Форматы: образов дисков и памяти, трасс
\item Автоматизация и расширяемость с помощью динамических языков: Perl, Python, Lua\dots
\item Автоматическая работа без человеческого вмешательства: без GUI, для регулярного тестирования
\end{itemize}
\end{frame}

\begin{frame}{Структура симулятора}
\begin{itemize}
\item Симуляционное ядро
\item Интерфейс командной строки
\item Графический интерфейс
\item Интерпретатор скриптов
\item Публичный API
\end{itemize}
\end{frame}

\section{Итоги}

\begin{frame}{Итоги}
\begin{itemize}
\item Точность
\item Скорость
\item Расширяемость
\end{itemize}
\end{frame}

\begin{frame}{На следующей лекции}
\begin{itemize}
\item Интерпретация
\end{itemize}
\end{frame}

% \section{Литература}

% \begin{frame}[allowframebreaks]{Литература}
% \begin{thebibliography}{99}
    % \bibitem{simbook} Программное моделирование вычислительных систем (лекции). \url{http://atakua.doesntexist.org/public/archive/simcourse/simulation-lectures-latest.pdf}

    % \bibitem{practicum} Лабораторный практикум по программному моделированию.  \url{http://atakua.doesntexist.org/public/archive/simcourse/simulation-practicum-latest.pdf}

% \end{thebibliography}
% \end{frame}


% The final "thank you" frame
\begin{frame}

{\huge{Спасибо за внимание!}\par}

\vfill

Слайды и материалы курса доступны по адресу \url{http://is.gd/ivuboc} % http://atakua.doesntexist.org/wordpress/simulation-course-russian/

\vfill

\tiny{\textit{Замечание}: все торговые марки и логотипы, использованные в данном материале, являются собственностью их владельцев. Представленная точка зрения отражает личное мнение автора.
%Материалы доступны по лицензии Creative Commons Attribution-ShareAlike (Атрибуция — С сохранением условий) 4.0 весь мир (в т.ч. Россия и др.). Чтобы ознакомиться с экземпляром этой лицензии, посетите \url{http://creativecommons.org/licenses/by-sa/4.0/}
}

\end{frame}


\end{document}

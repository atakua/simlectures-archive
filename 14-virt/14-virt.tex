% Compile with XeLaTeX, TeXLive 2013 or more recent
\documentclass{beamer}

% Base packages
\usepackage{fontspec}
\usepackage{xunicode}
\usepackage{xltxtra}

\usepackage{amsfonts}
\usepackage{amsmath}
\usepackage{longtable}
\usepackage{csquotes}
\usepackage{standalone}

% Setup fonts
\newfontfamily\russianfont{CMU Serif}
\setromanfont{CMU Serif}
\setsansfont{CMU Sans Serif}
\setmonofont{CMU Typewriter Text}

% Setup Russian hyphenation. NOTE: this declaration *must* come after fontspec's font declarations,
% or a mysterious (but harmless in other respects) error "Improper `at' size (0.0pt), replaced by 10pt." would appear.
\usepackage{polyglossia}
\defaultfontfeatures{Scale=MatchLowercase, Mapping=tex-text}

\setdefaultlanguage[spelling=modern]{russian} % for polyglossia
\setotherlanguage{english} % for polyglossia

% Vector drawings
\usepackage{tikz}
\usetikzlibrary{shapes, calc, arrows, fit, positioning, decorations, patterns, decorations.pathreplacing, chains, snakes}
\usepackage[siunitx]{circuitikz}

% Be able to insert hyperlinks
\usepackage{hyperref}
\hypersetup{colorlinks=true, linkcolor=black, filecolor=black, citecolor=black, urlcolor=blue , pdfauthor=Grigory Rechistov <grigory.rechistov@phystech.edu>, pdftitle=Курс «Программное моделирование вычислительных систем»}
% \usepackage{url}

% Misc optional packages
\usepackage{underscore}
\usepackage{amsthm}

% A new command to mark not done places
\newcommand{\todo}[1][Напиши меня]{{\color{red}TODO\ #1}}

\title{Современная виртуализация}
\subtitle{Курс «Программное моделирование вычислительных систем»}
\subject{Курс «Программное моделирование вычислительных систем»}

\author[]{Григорий Речистов \\ \small{\href{mailto:grigory.rechistov@phystech.edu}{grigory.rechistov@phystech.edu}}}
\date{\today}
\pgfdeclareimage[height=0.5cm]{ilab-logo}{../ilab-noletters.png}
\logo{\pgfuseimage{ilab-logo}}

\typeout{Copyright 2015 Grigory Rechistov}

\usetheme{Berlin}
\setbeamertemplate{navigation symbols}{}%remove navigation symbols

\begin{document}

\begin{frame}
\titlepage
\end{frame}

\begin{frame}
\tableofcontents
\end{frame}


\begin{frame}{На прошлых лекциях}
\begin{itemize}
\item Симуляция —  моделирование системы через имитацию внешне видимых эффектов, возникающих при взаимодействии с ней
\item Эмуляция — моделирование системы через имитацию внутренней структуры и процессов, происходящих внутри её подсистем
\item Виртуализация — обеспечение эффективной изоляции нескольких систем друг от друга при одновременном и прозрачном доступе к ресурсам нижележащей системы
\end{itemize}

\end{frame}

% \begin{frame}{Вопросы}
% \begin{itemize}
% \item Что такое «каузальность»?\pause
% \item ?\pause
% \item Чем GVT?
% \end{itemize}
% \end{frame}

\begin{frame}{Связь виртуализации и симуляции}

\centering
\input{../../simbook/metoda/drawings/simulation-levels}

\vfill

\input{../../simbook/metoda/drawings/multiplexing}

\end{frame}

\begin{frame}{История}
\begin{itemize}
\item IBM VM — 1960 гг. \cite{goldberg}
\item VMware Workstation — 1998 г.
\item Intel VT-x, VT-i — 2005 г.
\item 
\end{itemize}


\end{frame}

\section{Классические условия}

\begin{frame}{Необходимые свойства}

\begin{enumerate}
\item Изоляция — каждая виртуальная машина должна иметь доступ только к тем ресурсам, которые были ей назначены
\item Эквивалентность — любая программа, исполняемая под управлением ВМ, должна демонстрировать поведение, полностью идентичное реальной системе, за исключением эффектов, связанных с объёмами ресурсов
\item Эффективность — «статистически преобладающее подмножество инструкций виртуального процессора должно исполняться напрямую хозяйским процессором, без вмешательства монитора ВМ»
\end{enumerate}

\end{frame}

\begin{frame}{Модель}

\begin{itemize}
\item Один процессор, исполняющий инструкции
\item Состояние: (M, P, R)
\item Два режима M: u и s
\item Указатель текущей инструкции P
\item Границы сегмента памяти R (l, b)
\item Оперативная память: линейная E с ячейками E[n]
\end{itemize}

\end{frame}

\begin{frame}{События ловушки (trap)}
\begin{itemize}
\item Вызванные попыткой изменить состояние
процессора (потока управления)
\item Вызванные механизмом защиты памяти (ловушка з.п.)

\item E[0] ← (M1,P1,R1)
\item (M2,P2,R2) ← E[1]
\end{itemize}

\end{frame}

\begin{frame}{Инструкции}
\begin{itemize}
    \item Привилегированные (privileged). Исполнение с M=u всегда вызывает ловушку потока управления
    \item Служебные (sensitive)
    \begin{itemize}
        \item Инструкции, исполнение которых закончилось без ловушки защиты памяти и вызвало изменение M и/или R
        \item Инструкции, поведение которых в случаях, когда они не вызывают ловушку защиты памяти, зависит или от режима M, или от значения R
    \end{itemize}
    \item Безвредные (innocuous) — не служебные
\end{itemize}
\end{frame}

\begin{frame}[shrink=20]{Достаточное условие}

Множество служебных инструкций является подмножеством привилегированных инструкций

\centering
\input{../../simbook/metoda/drawings/vm-sufficient-condition} 
\end{frame}

\begin{frame}{Построение}
\begin{itemize}
    \item Программы ВМ исполняют безобидные инструкции напрямую
    \item Служебные инструкции вызывают ловушку → переход в монитор, который их эмулирует
    \item Привилегированные инструкции (ОС внутри ВМ) → ловушка
\end{itemize}

\centering$\Downarrow$

\begin{enumerate}
    \item Изоляция
    \item Эквивалентность
    \item Эффективность
\end{enumerate}
\end{frame}

\section{Современные корректировки}

\begin{frame}{Что не упомянуто в условии Г. и П.}

\begin{itemize}
    \item Сложные схемы трансляции адресов
    \item Периферия
    \item Многопроцессорные системы
\end{itemize}
\end{frame}

\begin{frame}[shrink=20]{Трансляция адресов}
\centering
\input{../../simbook/metoda/drawings/two-level-translation} 
\end{frame}

\begin{frame}{Виртуализация TLB}
\begin{center}
\begin{tabular}{rrr}
Виртуальный адрес & Физический адрес & Тэг\\
0x11112222000 &  0x22220000 & VM1\\
0x11112222000 &  0x11110000 & VM2\\
0x44443333000 &  0x55554000 & MON\\
0xabcd9876000 &  0x00001000 & VM1\\
0xabcd9876000 &  0x11111000 & VM3
 \end{tabular}
\end{center}
\end{frame}

\begin{frame}{Периферийные устройства}
\begin{itemize}
    \item Кому доставлять прерывание?
    \item Что делать, если прерывания внутри ВМ запрещены?
\end{itemize}

\vfill
\centering
\begin{tikzpicture}[>=latex, font=\small, align=center]
    \node[draw] (dev) {Устройство};

    \node[draw, rounded corners, right=2cm of dev, text width=2.2cm] (vmm) {Монитор ВМ};
    \node[draw, rounded corners, below = 0.5 cm of vmm, text width=2.2cm] (vm1) {ВМ 1};
    \node[draw, rounded corners, below = 0.5cm of vm1, text width=2.2cm] (vm2) {ВМ 2};

    \node[draw, left=0.5cm of vm1, cloud] (sel) {?};

    \draw[->] (dev) |- (sel);
    \draw[->] (sel) -- (vmm);
    \draw[->] (sel) -- (vm1);
    \draw[->] (sel) -- (vm2);
\end{tikzpicture}

\end{frame}

\begin{frame}{Консервативный подход}
\begin{itemize}
    \item Все прерывания доставляются монитору
    \item Монитор «впрыскивает» их в ВМ
    \item Повышенная задержка доставки прерываний
\end{itemize}

\vfill
\centering
\begin{tikzpicture}[>=latex, font=\small, align=center]
    \node[draw] (dev) {Устройство};

    \node[draw, rounded corners, right=2cm of dev, text width=2.2cm] (vmm) {Монитор ВМ};
    \node[draw, rounded corners, below = 0.5 cm of vmm, text width=2.2cm] (vm1) {ВМ 1};
    \node[draw, rounded corners, below = 0.5cm of vm1, text width=2.2cm] (vm2) {ВМ 2};

    \draw[->] (dev) -- (vmm);
    \draw[->] (vmm) -- (vm1);
\end{tikzpicture}

\end{frame}

\begin{frame}{Аппаратная поддержка}
Аппаратура обеспечивает доставку выбранных прерываний в текущую ВМ, остальные — в монитор

\vfill
\centering
\begin{tikzpicture}[>=latex, font=\small, align=center]
    \node[draw] (dev) {Устройство};

    \node[draw, rounded corners, right=2.5cm of dev, text width=2.2cm] (vmm) {Монитор ВМ};
    \node[draw, rounded corners, below = 0.5 cm of vmm, text width=2.2cm] (vm1) {ВМ 1};
    \node[draw, rounded corners, below = 0.5cm of vm1, text width=2.2cm] (vm2) {В М 2};

    \node[draw, left=1.5cm of vm1, trapezium, shape border rotate=270] (sel) {HW};

    \draw[->] (dev) |- (sel);
    \draw[->] (sel) -- (vmm.west) node[above, midway, font=\tiny, xshift=-0.25cm] {Остальные\\прерывания};
    \draw[->] (sel) -- (vm1) node[below, midway, font=\tiny ] {Разрешённые\\прерывания};
\end{tikzpicture}


\end{frame}


\begin{frame}{Многопроцессорность}
\begin{itemize}
    \item Планировка исполнения $N$ виртуальных процессоров на $M$ физических, $N \geqslant M$
    \begin{itemize}
        \item Справедливая (fair)
        \item Эффективная — характерные длительности синхронизационных процессов внутри ВМ должны быть близки к наблюдаемым на реальной аппаратуре
    \end{itemize}
\end{itemize}

\begin{itemize}
    \item Проблема вытеснения потоков, заблокировавших ресурсы (lock holder preemption)
    \item Монитору необходимо детектировать новый класс гостевых инструкций — синхронизационные примитивы (атомарные инструкции)
\end{itemize}

\end{frame}

\begin{frame}[allowframebreaks]{Рекомендуемая литература}
\begin{thebibliography}{99}
\bibitem{goldberg} Popek Gerald J., Goldberg Robert P. Formal requirements for virtualizable third generation architectures // Communications of the ACM. V. 17. \#7. 1974.
\bibitem{vtx} Leung, F. [et al.] Intel® Virtualization Technology // ITJ Vol 10 Issue 3 — 2006.
\bibitem{mpr} Harlan McGhan. The gHost in the Machine: Parts 1,2,3 // Microprocessor Report. 2007. \url{http://mpronline.com}
\bibitem{virt-locks} Jiannan Ouyang and John R. Lange. Preemptable ticket spinlocks: improving consolidated performance in the cloud — 2013 \url{https://labs.vmware.com/vee2013/docs/p191.pdf}

\end{thebibliography}
\end{frame}


\begin{frame}{На следующей лекции}
\centering


\end{frame}


\begin{frame}

{\huge{Спасибо за внимание!}\par}

\vfill

Слайды и материалы курса доступны по адресу \url{http://is.gd/ivuboc} % http://atakua.doesntexist.org/wordpress/simulation-course-russian/

\vfill

\tiny{\textit{Замечание}: все торговые марки и логотипы, использованные в данном материале, являются собственностью их владельцев. Представленная здесь точка зрения отражает личное мнение автора, не выступающего от лица какой-либо организации.}

\end{frame}

\end{document}

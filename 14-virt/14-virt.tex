% Compile with XeLaTeX, TeXLive 2013 or more recent
\documentclass{beamer}

% Base packages
\usepackage{fontspec}
\usepackage{xunicode}
\usepackage{xltxtra}

\usepackage{amsfonts}
\usepackage{amsmath}
\usepackage{longtable}
\usepackage{csquotes}
\usepackage{standalone}

% Setup fonts
\newfontfamily\russianfont{CMU Serif}
\setromanfont{CMU Serif}
\setsansfont{CMU Sans Serif}
\setmonofont{CMU Typewriter Text}

% Setup Russian hyphenation. NOTE: this declaration *must* come after fontspec's font declarations,
% or a mysterious (but harmless in other respects) error "Improper `at' size (0.0pt), replaced by 10pt." would appear.
\usepackage{polyglossia}
\defaultfontfeatures{Scale=MatchLowercase, Mapping=tex-text}

\setdefaultlanguage[spelling=modern]{russian} % for polyglossia
\setotherlanguage{english} % for polyglossia

% Vector drawings
\usepackage{tikz}
\usetikzlibrary{shapes, calc, arrows, fit, positioning, decorations, patterns, decorations.pathreplacing, chains, snakes}
\usepackage[siunitx]{circuitikz}

% Be able to insert hyperlinks
\usepackage{hyperref}
\hypersetup{colorlinks=true, linkcolor=black, filecolor=black, citecolor=black, urlcolor=blue , pdfauthor=Grigory Rechistov <grigory.rechistov@phystech.edu>, pdftitle=Курс «Программное моделирование вычислительных систем»}
% \usepackage{url}

% Misc optional packages
\usepackage{underscore}
\usepackage{amsthm}

% A new command to mark not done places
\newcommand{\todo}[1][Напиши меня]{{\color{red}TODO\ #1}}

\title{Современная виртуализация}
\subtitle{Курс «Программное моделирование вычислительных систем»}
\subject{Курс «Программное моделирование вычислительных систем»}

\author[]{Григорий Речистов \\ \small{\href{mailto:grigory.rechistov@phystech.edu}{grigory.rechistov@phystech.edu}}}
\date{\today}
\pgfdeclareimage[height=0.5cm]{ilab-logo}{../ilab-noletters.png}
\logo{\pgfuseimage{ilab-logo}}

\typeout{Copyright 2015 Grigory Rechistov}

\usetheme{Berlin}
\setbeamertemplate{navigation symbols}{}%remove navigation symbols

\begin{document}

\begin{frame}
\titlepage
\end{frame}

\begin{frame}
\tableofcontents
\end{frame}

\section{Обзор}

\begin{frame}{На прошлых лекциях}
\begin{itemize}
\item 
\item 
\end{itemize}

\end{frame}

\begin{frame}{Вопросы}
\begin{itemize}
\item ?\pause
\item ?\pause
\item Time Warp?
\end{itemize}

\end{frame}

\begin{frame}{Связь виртуализации и симуляции}

\end{frame}

\begin{frame}{}
•IBM System/360 – 1960 гг.

\end{frame}

\section{Классические условия}

\begin{frame}{Необходимые свойства}

•Изоляция — каждая виртуальная машина должна иметь
доступ только к тем ресурсам, которые были ей
назначены.
•Эквивалентность — любая программа, исполняемая под
управлением ВМ, должна демонстрировать поведение,
полностью идентичное реальной системе, за исключением
эффектов,
•Эффективность — «статистически преобладающее
подмножество инструкций виртуального процессора
должно исполняться напрямую хозяйским процессором,
без вмешательства монитора ВМ»

\end{frame}

\begin{frame}{Модель}

Один процессор, исполняющий
инструкции
• Состояние: (M, P, R)
• Два режима M: u и s
• Указатель текущей инструкции P
• Границы сегмента памяти R (l,b)
Оперативная память
• Линейная E с ячейками E[n]

\end{frame}

\begin{frame}{}
События ловушки (trap)
•Вызванные попыткой изменить состояние
процессора (потока управления)
•Вызванные механизмом защиты памяти
(ловушка з.п.)
•E[0] ← (M1,P1,R1)
•(M2,P2,R2) ← E[1]

\end{frame}

\begin{frame}{}
Инструкции
•Привилегированные (privileged). Исполнение с
M=u всегда вызывает ловушку потока
управления.
•Служебные (sensitive)
• Инструкции, исполнение которых
закончилось без ловушки защиты памяти и
вызвало изменение M и/или R.
• Инструкции, поведение которых в случаях,
когда они не вызывают ловушку защиты
памяти, зависит или от режима M, или от
значения R.
•Безвредные — не служебные
Лаборатория суперкомпьютерных технологий для биомедицины, фармакологии и
\end{frame}

\begin{frame}[shrink=20]{Достаточное условие}

Множество служебных инструкций является
подмножеством привилегированных инструкций

\centering
\input{../../simbook/metoda/drawings/vm-sufficient-condition} 
\end{frame}

\begin{frame}{}
Почему это так
•Программы исполняют безобидные инструкции напрямую
•Служебные инструкции вызывают ловушку → переход в
монитор, который их эмулирует
•Привилегированные инструкции (ОС в ВМ) → ловушка
•Изоляция
•Эквивалентность
•Эффективность

\end{frame}

\section{Современные корректировки}

\begin{frame}{}
Что не упомянуто в условии Г. и П.
•Сложные схемы трансляции адресов
•Периферия
•Многопроцессорные системы

\end{frame}

\begin{frame}{Трансляция адресов}
\centering
\input{../../simbook/metoda/drawings/two-level-translation} 
\end{frame}

\begin{frame}{TLB}

\end{frame}

\begin{frame}{Периферийные устройства}
•Кому доставлять
прерывание?
•Что делать, если
прерывания внутри ВМ
запрещены?

\end{frame}

\begin{frame}{Консервативный подход}
•Все прерывания доставляется
монитору
•Монитор «впрыскивает» их в ВМ
•Повышенная латентность
доставки прерываний

\end{frame}

\begin{frame}{Аппаратная поддержка}
•Аппаратура поддерживает
выборочную доставку
прерываний напрямую в ВМ

\end{frame}


\begin{frame}{Многопроцессорность}

•Планировка исполнения N виртуальных процессоров на M
физических, $N \geqslant M$
• Справедливая (fairness)
• Эффективная — характерные длительности
синхронизационных процессов внутри ВМ должны
быть близки к наблюдаемым на реальной аппаратуре
•Проблема вытеснения потоков, заблокировавших ресурсы
(lock holder preemption)
• Монитору необходимо детектировать новый класс
гостевых инструкций — синхронизационные
примитивы (атомарные)

\end{frame}


\begin{frame}[allowframebreaks]{Рекомендуемая литература}
\begin{thebibliography}{99}
\bibitem{vtx} Intel VT-x
\bibitem{mpr} Harlan McGhan. The gHost in the Machine: Parts 1,2,3 // Microprocessor Report. 2007. \url{http://mpronline.com}
\bibitem{goldberg} Popek Gerald J., Goldberg Robert P. Formal requirements for virtualizable third generation
architectures // Communications of the ACM. V. 17. \#7. 1974.


\end{thebibliography}
\end{frame}


\begin{frame}{На следующей лекции}
\centering

\end{frame}


\begin{frame}

{\huge{Спасибо за внимание!}\par}

\vfill

Слайды и материалы курса доступны по адресу \url{http://is.gd/ivuboc} % http://atakua.doesntexist.org/wordpress/simulation-course-russian/

\vfill

\tiny{\textit{Замечание}: все торговые марки и логотипы, использованные в данном материале, являются собственностью их владельцев. Представленная здесь точка зрения отражает личное мнение автора, не выступающего от лица какой-либо организации.}

\end{frame}

\end{document}

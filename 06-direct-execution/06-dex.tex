% Compile with XeLaTeX, TeXLive 2013 or more recent
\documentclass{beamer}

% Base packages
\usepackage{fontspec}
\usepackage{xunicode}
\usepackage{xltxtra}

\usepackage{amsfonts}
\usepackage{amsmath}
\usepackage{longtable}
\usepackage{csquotes}
\usepackage{standalone}
\usepackage{bytefield}

% Setup fonts
\newfontfamily\russianfont{CMU Serif}
\setromanfont{CMU Serif}
\setsansfont{CMU Sans Serif}
\setmonofont{CMU Typewriter Text}

% Setup Russian hyphenation. NOTE: this declaration *must* come after fontspec's font declarations,
% or a mysterious (but harmless in other respects) error "Improper `at' size (0.0pt), replaced by 10pt." would appear.
\usepackage{polyglossia}
\defaultfontfeatures{Scale=MatchLowercase, Mapping=tex-text}

\setdefaultlanguage[spelling=modern]{russian} % for polyglossia
\setotherlanguage{english} % for polyglossia

% Vector drawings
\usepackage{tikz}
% \usetikzlibrary{shapes, calc, arrows, fit, positioning, decorations, patterns, decorations.pathreplacing, chains, snakes}
\usetikzlibrary{shapes, calc, arrows, decorations.markings, decorations.pathreplacing, decorations.pathmorphing, decorations, patterns, chains, snakes, backgrounds, positioning, fit, shadows}
% \usepackage[siunitx]{circuitikz}

% Be able to insert hyperlinks
\usepackage{hyperref}
\hypersetup{colorlinks=true, linkcolor=black, filecolor=black, citecolor=black, urlcolor=blue , pdfauthor=Grigory Rechistov <grigory.rechistov@phystech.edu>, pdftitle=Курс «Программное моделирование вычислительных систем»}
% \usepackage{url}

% Misc optional packages
\usepackage{underscore}
\usepackage{amsthm}

% A new command to mark not done places
\newcommand{\todo}[1][Напиши меня]{{\color{red}TODO\ #1}}
\newcommand{\abbr}{\textit{англ.}\ }

\title{Прямое исполнение}
\subtitle{Курс «Программное моделирование вычислительных систем»}
\subject{Курс «Программное моделирование вычислительных систем»}

\author[]{Григорий Речистов \\ \small{\href{mailto:grigory.rechistov@phystech.edu}{grigory.rechistov@phystech.edu}}}
\date{\today}
\pgfdeclareimage[height=0.5cm]{ilab-logo}{../ilab-noletters.png}
\logo{\pgfuseimage{ilab-logo}}

\typeout{Copyright 2015 Grigory Rechistov}

\usetheme{Berlin}
\setbeamertemplate{navigation symbols}{}%remove navigation symbols

\begin{document}

\begin{frame}
    \maketitle
\end{frame}

\begin{frame}
    \tableofcontents
\end{frame}


\begin{frame}{На прошлой лекции}
\begin{itemize}
\item Интерпретаторы — медленная шутка
\item Двоичная трансляция быстрее, потому что вычисляет «меньше»
\end{itemize}
\end{frame}

\begin{frame}{Вопросы}
\begin{itemize}
\item ЯВО → маш. код — компиляция. Маш.~код → маш.~код — ДТ. А что такое маш.~код → ЯВО?\pause{} Декомпиляция.\pause
\item А ЯВО → ЯВО?\pause Source-level компилятор\pause
\item В каких случаях ДТ будет медленнее интерпретации на одной и той же гостевой программе? \pause Если программа полна SMC.
\end{itemize}

\end{frame}

% «»

\section{Прямое исполнение}

\begin{frame}{Когда применимо прямое исполнение}
\begin{itemize}
\item Когда гостевая ISA совпадает с хозяйской
\item Ну или почти совпадает
\end{itemize}

\end{frame}


\begin{frame}{Алгоритм}
\begin{itemize}

\item вв
\end{itemize}
\end{frame}

\begin{frame}{Почему это не будет работать}
\begin{itemize}
\item Не полностью совпадающие ISA

\item Различное положение внешних ресурсов
(устройств и памяти)

\item Привилегированность инструкций

\item Необходимость изоляции симулятора от
обнаружения и разрушения гостем

\end{itemize}
\end{frame}

\begin{frame}{Почему это не будет работать}


add \%r1, \%r2

mul \$10, \%r3

div \%r4, \%r5 Отсутствующая в хозяине инструкция

ld (0xa000), \%r10 Другое расположение в памяти

st \%r10, (\%r11)

sub \%r11, \%r1

mov \$16, \%r13

mov \%r13, \%cr0 Привилегированные инструкции

trap \$32


\end{frame}

\section{Предпросмотр}

\begin{frame}{Предпросмотр кода}

\end{frame}


\begin{frame}{Заплатки и заглушки}

\end{frame}

\begin{frame}{Двоичная инструментация}
•••Общее название методики
исследования и модификации
приложений
Pin http://pintool.org
DynamoRIO http://dynamorio.org/

\end{frame}


\begin{frame}[fragile]{Сложности DEX}

•••••Необходимость предпросмотра негативно
влияет на производительность симуляции
Необходимость контролировать SMC
Переменная длина инструкций усложняет
stubbing/patching
Необходимо контролировать время
исполнения гостя
• А как это делается в многозадачных
вытесняющих ОС?
Для DEX оптимально иметь аппаратную
поддержку на хозяине

\end{frame}

\begin{frame}{Спектр симуляционных подходов}

\end{frame}

\section{Коробка передач}

\begin{frame}{Коробка передач}

\end{frame}


\begin{frame}{Динамическое переключение режимов}
+Оптимальное использование
лучших сторон каждого из подходов
- Необходимость разработки
фактически нескольких симуляторов

\end{frame}

\begin{frame}{Итоги}

Наивное прямое исполнение
Заплатки и заглушки
DEX с аппаратной поддержкой
Переключение режимов симуляции
• Условия на переходы
    
\end{frame}

\begin{frame}{Оптимизации}
\centering
\input{./../../simbook/metoda/drawings/bt-optimization}

\end{frame}

\section{Заключение}

\begin{frame}{Итоги}
\begin{itemize}

\item Наивное прямое исполнение

\item Заплатки и заглушки

\item DEX с аппаратной поддержкой

\item Переключение режимов симуляции

\item Условия на переходы

\end{itemize}
\end{frame}


\begin{frame}[allowframebreaks]{Литература}
\begin{thebibliography}{99}
    \bibitem{vms} F. Leung, G. Neiger, D. Rodgers, A. Santoni, R. Uhlig. Intel®
    Virtualization Technology // Intel Technology Journal No10 (03
Aug 2006).
http://www.intel.com/technology/itj/2006/v10i3/


Matias Zabaljauregui. Hardware Assisted Virtualization
Intel Virtualization Technology. 2008.
http://lib.mipt.ru/book/283035/


\end{thebibliography}
\end{frame}


\begin{frame}{На следующей лекции}
\centering

Симуляция периферийных устройств и полной платформы

\end{frame}

% The final "thank you" frame
\begin{frame}

{\huge{Спасибо за внимание!}\par}

\vfill

Слайды и материалы курса доступны по адресу \url{http://is.gd/ivuboc} % http://atakua.doesntexist.org/wordpress/simulation-course-russian/

\vfill

\tiny{\textit{Замечание}: все торговые марки и логотипы, использованные в данном материале, являются собственностью их владельцев. Представленная здесь точка зрения отражает личное мнение автора, не выступающего от лица какой-либо организации.}

\end{frame}

\end{document}


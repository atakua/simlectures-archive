% Compile with XeLaTeX, TeXLive 2013 or more recent
\documentclass{beamer}

% Base packages
\usepackage{fontspec}
\usepackage{xunicode}
\usepackage{xltxtra}

\usepackage{amsfonts}
\usepackage{amsmath}
\usepackage{longtable}
\usepackage{csquotes}
\usepackage{standalone}
\usepackage{bytefield}

% Setup fonts
\newfontfamily\russianfont{CMU Serif}
\setromanfont{CMU Serif}
\setsansfont{CMU Sans Serif}
\setmonofont{CMU Typewriter Text}

% Setup Russian hyphenation. NOTE: this declaration *must* come after fontspec's font declarations,
% or a mysterious (but harmless in other respects) error "Improper `at' size (0.0pt), replaced by 10pt." would appear.
\usepackage{polyglossia}
\defaultfontfeatures{Scale=MatchLowercase, Mapping=tex-text}

\setdefaultlanguage[spelling=modern]{russian} % for polyglossia
\setotherlanguage{english} % for polyglossia

% Vector drawings
\usepackage{tikz}
% \usetikzlibrary{shapes, calc, arrows, fit, positioning, decorations, patterns, decorations.pathreplacing, chains, snakes}
\usetikzlibrary{shapes, calc, arrows, decorations.markings, decorations.pathreplacing, decorations.pathmorphing, decorations, patterns, chains, snakes, backgrounds, positioning, fit, shadows}
% \usepackage[siunitx]{circuitikz}

% Be able to insert hyperlinks
\usepackage{hyperref}
\hypersetup{colorlinks=true, linkcolor=black, filecolor=black, citecolor=black, urlcolor=blue , pdfauthor=Grigory Rechistov <grigory.rechistov@phystech.edu>, pdftitle=Курс «Программное моделирование вычислительных систем»}
% \usepackage{url}

% Misc optional packages
\usepackage{underscore}
\usepackage{amsthm}

% A new command to mark not done places
\newcommand{\todo}[1][Напиши меня]{{\color{red}TODO\ #1}}
\newcommand{\abbr}{\textit{англ.}\ }

\title{Прямое исполнение}
\subtitle{Курс «Программное моделирование вычислительных систем»}
\subject{Курс «Программное моделирование вычислительных систем»}

\author[]{Григорий Речистов \\ \small{\href{mailto:grigory.rechistov@phystech.edu}{grigory.rechistov@phystech.edu}}}
\date{\today}
\pgfdeclareimage[height=0.5cm]{ilab-logo}{../ilab-noletters.png}
\logo{\pgfuseimage{ilab-logo}}

\typeout{Copyright 2015 Grigory Rechistov}

\usetheme{Berlin}
\setbeamertemplate{navigation symbols}{}%remove navigation symbols

\begin{document}

\begin{frame}
    \maketitle
\end{frame}

\begin{frame}
    \tableofcontents
\end{frame}


\begin{frame}{На прошлой лекции}
\begin{itemize}
\item Интерпретаторы — медленная шутка
\item Двоичная трансляция быстрее, потому что вычисляет «меньше»
\end{itemize}
\end{frame}

\begin{frame}{Вопросы}
\begin{itemize}
\item ЯВО → маш. код — компиляция. Маш.~код → маш.~код — ДТ. А что такое маш.~код → ЯВО?\pause{} Декомпиляция\pause
\item А ЯВО → ЯВО?\pause{} Source-level компилятор\pause
\item В каких случаях ДТ будет медленнее интерпретации на одной и той же гостевой программе? \pause Если часто происходят ретрансляции, например, программа полна SMC
\end{itemize}

\end{frame}

% «»

\section{Прямое исполнение}

\begin{frame}{Когда применимо прямое исполнение}
Идея: не симулировать код вообще! Direct execution, DEX
\begin{itemize}
\item Когда гостевая ISA совпадает с хозяйской
\item Ну или почти совпадает
\end{itemize}

\end{frame}

\begin{frame}[fragile]{Алгоритм}
\begin{verbatim}
execute() {
    save_host_ctx();
    set_guest_ctx();
    setjmp(back);
    goto guest_start_ip;
    back: restore_host_ctx();
}
\end{verbatim}
\end{frame}

\begin{frame}{Почему это не будет работать}
\begin{itemize}
\item Не полностью совпадающие ISA
\item Различное положение внешних ресурсов (устройств и памяти)
\item Привилегированность некоторых инструкций
\item Необходимость изоляции симулятора от обнаружения и разрушения гостем
\end{itemize}
\end{frame}

\begin{frame}{Почему это не будет работать}

\texttt{add \%r1, \%r2} \hfill OK

\texttt{mul \$10, \%r3}

\texttt{sub \%r11, \%r1}

\texttt{mov \$16, \%r13}

\medskip

\texttt{\textcolor{red}{div} \%r4, \%r5} \hfill Отсутствующая в хозяине инструкция

\medskip

\texttt{ld (\textcolor{red}{0xa000}), \%r10} \hfill Другое расположение в памяти

\texttt{st \%r10, (\%r11)}

\medskip

\texttt{mov \%r13, \textcolor{red}{\%cr0}} \hfill Привилегированные инструкции

\texttt{\textcolor{red}{trap} \$32}

\end{frame}

\section{Предпросмотр}

\begin{frame}{Предпросмотр кода}

\begin{itemize}
\item Инспектирование гостевого кода на предмет «опасных» инструкций перед исполнением
\item Замена части инструкций на контролируемые — \textit{инструментация}
\item «Почти» двоичная трансляция
\end{itemize}

\end{frame}


\begin{frame}{Заплатки и заглушки}
\begin{tabular}{ll}
Исходный код                    &    Код после инструментации \\
\texttt{add \%r1, \%r2}         &    \texttt{add \%r1, \%r2} \\
\texttt{mul \$10, \%r3}         &    \texttt{mul \$10, \%r3} \\
\texttt{div \%r4, \%r5}         &    \texttt{trap \$255} \\
\texttt{ld (0xa000), \%r10}     &    \texttt{ld (0xb000), \%r10} \\
\texttt{st \%r10, (\%r11)}      &    \texttt{st \%r10, (\%r11)} \\
\texttt{sub \%r11, \%r1}        &    \texttt{sub \%r11, \%r1} \\
\texttt{mov \$16, \%r13}        &    \texttt{mov \$16, \%r13} \\
\texttt{mov \%r13, \%cr0}       &    \texttt{trap \$255} \\
\texttt{trap \$32}              &    \texttt{trap \$255} \\
\end{tabular}

\vfill
stub — заглушка, patch — заплатка

\end{frame}

\begin{frame}{Двоичная инструментация}
Общее название для методик исследования и модификации приложений

\begin{itemize}
\item Pin \url{http://pintool.org}
\item DynamoRIO \url{http://dynamorio.org}
\end{itemize}

\end{frame}

\begin{frame}[fragile]{Сложности DEX}

\begin{itemize}
\item Необходимость предпросмотра негативно влияет на производительность симуляции
\item Необходимость контролировать возможность самомодификации кода
\item Переменная длина инструкций усложняет установку заплаток и заглушек
\item Необходимость гарантировать, что управление вернётся из гостя в симулятор
\end{itemize}
\end{frame}

\section{Аппаратная поддержка}

\begin{frame}{Аппаратная поддержка прямого исполнения}
\begin{itemize}
    \item Набор интерфейсов аппаратуры хозяйской системы, предназначенных для упрощения DEX
    \item Аппаратная поддержка типичных операций: загрузка гостевого состояния, контроль за исполнением инструкций и доступов к ресурсам, обработка внешних и внутренних исключений
    \item Аппаратная виртуализация: тема отдельной лекции
\end{itemize}
\end{frame}

\begin{frame}{Плюсы и минусы}
\begin{itemize}
    \item Прямое исполнение большинства инструкций $\Rightarrow$ самый быстрый способ симуляции
    \item Упрощение модели, уменьшение объёма кода симулятора
\end{itemize}

\begin{itemize}
    \item Необходима аппаратная поддержка
    \item Надо писать код для ядра ОС
    \item Медленное переключение между режимами %может снизить производительность
    \item Работает только при совпадении архитектур
    \item Не все режимы процессора могут быть симулированы
\end{itemize}

Предпочтительный подход: модуль ядра для частых операций и пользовательское приложение для всего остального

\end{frame}


\section{Коробка передач}
\begin{frame}{Спектр симуляционных подходов}
\centering
\begin{tikzpicture}[>=latex,
                    node distance=0.25cm, font=\small,
                    align=left,
                    start chain=going {at=(\tikzchainprevious),shift=(30:1.5)}]

\node[on chain] (n1) {Интерпретатор}; % переключаемый
% \node[on chain] {Интерпретатор сцепленный};
% \node[on chain] {Интерпретатор кэширующий};
\node[on chain] {Двоичная трансляция};
\node[on chain] {Прямое исполнение};
\node[on chain] (n5) {Реальная аппаратура};

\coordinate[right=0.75cm of n1] (a1);
\coordinate[right=0.75cm of n5] (a5);
\coordinate[left=0.75cm of n1] (b1);
\coordinate[left=0.75cm of n5] (b5);

\pause\draw[->] (a1) -- (a5) node[inner xsep=3pt, pos=0.7, right=0.25cm] {\scriptsize Скорость};
\pause\draw[->] (b5) -- (b1) node[inner xsep=3pt, pos=0.7, left=0.25cm] {\scriptsize Универсальность};


\end{tikzpicture}

%  Интерпретатор
%
%  Прямое
% ()
%  ()
%  ()
%
%  исполнение

\end{frame}


\begin{frame}{Коробка передач}
\centering

\input{./../../simbook/metoda/drawings/gearbox}

\end{frame}

\begin{frame}{Динамическое переключение режимов}
\begin{itemize}
    \item Оптимальное использование лучших сторон каждого из подходов
    \item Необходимость разработки фактически нескольких симуляторов
\end{itemize}
\end{frame}

\section{Заключение}
\begin{frame}{Итоги}
\begin{itemize}
\item Наивное прямое исполнение
\item Заплатки и заглушки
\item Аппаратная поддержка прямого исполнения
\item Переключение режимов симуляции и условия на переходы между ними
\end{itemize}

\end{frame}

\begin{frame}[allowframebreaks]{Литература}
\begin{thebibliography}{99}
    \bibitem{pinpapers} Pin - A Binary Instrumentation Tool - Papers \url{https://software.intel.com/en-us/articles/pin-a-binary-instrumentation-tool-papers}
    \bibitem{pin} Heidi Pan, Krste Asanovíc, Robert Cohn and Chi-Keung Luk. Controlling Program Execution through Binary Instrumentation \url{http://www.cs.berkeley.edu/~krste/papers/pin-wbia.pdf}
    \bibitem{vms} F. Leung, G. Neiger, D. Rodgers, A. Santoni, R. Uhlig. Intel Virtualization Technology: Hardware Support for Efficient Processor Virtualization // Intel Technology Journal 10 (03) Aug 2006

    {\scriptsize\url{http://atakua.doesntexist.org/public/archive/papers/Intel-VT-Hardware-Support-for-Efficient-Processor-Virtualization.pdf}}
\end{thebibliography}
\end{frame}


\begin{frame}{На следующей лекции}
\centering

Симуляция периферийных устройств и полной платформы

\end{frame}

% The final "thank you" frame
\begin{frame}

{\huge{Спасибо за внимание!}\par}

\vfill

Слайды и материалы курса доступны по адресу \url{http://is.gd/ivuboc} % http://atakua.doesntexist.org/wordpress/simulation-course-russian/

\vfill

\tiny{\textit{Замечание}: все торговые марки и логотипы, использованные в данном материале, являются собственностью их владельцев. Представленная здесь точка зрения отражает личное мнение автора, не выступающего от лица какой-либо организации.}

\end{frame}

\end{document}

